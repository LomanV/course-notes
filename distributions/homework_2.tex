\documentclass[12pt]{article}
\usepackage[top=1in, bottom=1in, left=1in, right=1in]{geometry}

\usepackage[onehalfspacing]{setspace}

\usepackage{amsmath, amssymb, amsthm}
\usepackage{enumerate, enumitem}
\usepackage{fancyhdr, graphicx, proof, comment, multicol}
\usepackage[none]{hyphenat}
\usepackage{dirtytalk}
\binoppenalty=\maxdimen
\relpenalty=\maxdimen

\usepackage{microtype}
\usepackage{mathpazo}
\usepackage{mdframed}
\usepackage{parskip}
\linespread{1.1}
\usepackage{graphicx}
\usepackage{subfig}

\usepackage{mathrsfs}
\usepackage{amsfonts}
\usepackage{amsmath}
\usepackage{amssymb}

\usepackage{mathtools}
\newcommand{\defeq}{\vcentcolon=}
\newcommand{\eqdef}{=\vcentcolon}

\newenvironment{statement}[1]
{\begin{mdframed}[linewidth=0.6pt]
        \textsc{Statement #1:}

}

\newenvironment{ex}[1]
{\begin{mdframed}[linewidth=0.6pt]
        \textsc{Exercice #1:}

}
    {\end{mdframed}}

\newcommand{\R}{\mathbf{R}}
\newcommand{\C}{\mathbf{C}}
\newcommand{\Z}{\mathbf{Z}}
\newcommand{\N}{\mathbf{N}}
\newcommand{\Q}{\mathbf{Q}}

\newcommand{\de}{\mathrm{d}}

\begin{document}
        \noindent
\textbf{MAT432 Distributions} \hfill \textbf{Vezin Lomàn}\\
\normalsize MAT432  \hfill Date de rendu: 11/03/2021\\

\begin{center}
\textbf{Exercices contrôle continu}
\end{center}
        
\textbf{Exercice 5} \quad \textit{f.} \; On montre l'égalité suivante \[
        -\partial_x^2 u_{n}(x) + x^2 u_n(x) = (2n + 1)u_{n}(x)
.\]

On part de $u_{n}(x) = C_n \cdot \exp(-\frac{x^2}{2})H_n(x)$ avec $C_n$ une constante multiplicative ne dépendant que de $n$ donnée dans le point \textit{e.}

Ainsi \[
        \partial_x u_n(x) = C_n\cdot(-x\exp(-\frac{x^2}{2})H_n(x) + \exp(-\frac{x^2}{2})H_n'(x))
,\] et \[
\partial_x^2 u_n(x) = C_n\cdot(-\exp(-\frac{x^2}{2})H_n(x) + x^2\exp(-\frac{x^2}{2})H_n(x) -2x\exp(-\frac{x^2}{2})H_n'(x) +\exp(-\frac{x^2}{2})H_n''(x))
.\]  

On obtient donc finalement en se souvenant de l'expression $u_n$ et après simplifications \[
        -\partial_x^2 u_n(x) +x^2u_n(x) = C_n\cdot(\exp(-\frac{x^2}{2})H_n(x) + 2x\exp(-\frac{x^2}{2})H_n'(x) + \exp(-\frac{x^2}{2})H_n''(x)') 
.\] 

On utilise à présent la relation sur $H_n''$ établie au point \textit{d}, ce qui donne après simplifications des termes en $H_n'(x)$ \[
        -\partial_x^2 u_n(x) +x^2u_n(x) = C_n\cdot(\exp(-\frac{x^2}{2})H_n(x) + 2n\exp(-\frac{x^2}{2})H_n(x)) 
.\] 

On factorise par $u_n(x)$ pour trouver finalement le résultat attendu \[
        \boxed{-\partial_x^2 u_n(x) +x^2u_n(x) = (2n+1)u_n(x).}
\]

\medskip

\textit{g.} On montre que la fonction $\psi$ donnée est bien continue en temps, $L^2$ en $x$.

Comme $\psi^{in}$ et $u_n$ sont des fonctions $L^2$, on a 
\begin{align*}
        |\int_{\R}\psi^{in}(y)u_n(y)\de y| & \le \int_{\R}|\psi^{in}(y)u_n(y)|\de y \\
                                           &\le (\int_{\R}\psi^{in}(y)^{2}\de y)^\frac{1}{2}\cdot(\int_{\R}u_n(y)^{2}\de y)^{\frac{1}{2}} \\
                                           &< +\infty
                                   ,\end{align*} où la deuxième inégalité est l'inégalité de Hölder. Le terme indépendant de $x, t$ est borné et donc $\psi$ est bien définie. Pour simplifier la rédaction on omettra dorénavant cette constante.

On montre la continuité en $t$. Pour tout $n \in \N$ la fonction \[
        f_n(t, x) \defeq \exp(-i\frac{2n+1}{2}t)u_n(x)
\] est continue en $t$ sur $\R$. De plus
\begin{align*}
        \sum_{n\in\N}|f_n(t, x)| &\le \sum_{n\in\N}|u_n(x)| \\
                                  &\le \frac{1}{\pi^{\frac{1}{4}}}\sum_{n\in\N}\frac{\exp(-x^2)^{(n)}}{\sqrt{2^{n}n!}} < +\infty
.\end{align*}

La série converge normalement indépendamment de $t \in \R$ donc uniformément, la continuité de chaque $f_n$ garantit celle de $\psi$. 

\medskip

On regarde l'intégrale \[
        \int_{\R}\psi(t, x)^{2}\de x
.\] Le développement du carré de la somme nous donne la somme des termes au carré plus la double somme des termes croisés. Par le point \textit{e.} nous savons que les $(u_n)_{n\in\N}$ forment une base de Hilbert de $L^2$, ils sont donc orthogonaux. La convergence normale nous permet de plus d'intervertir somme et intégrale. Les termes croisés sont donc tous nuls et il reste \[
\int_{\R}\psi(t, x)^{2}\de x = \sum_{n\in\N}\exp(-i(2n+1)t)\int_{\R}u_n(x)^{2}\de x
.\]  
On utilise la définition de $u_n$ pour trouver
\begin{align*}
        \int_{\R}\psi(t, x)^{2}\de x &= \sum_{n\in\N}\frac{\exp(-i(2n+1)t)}{2^{n}n!}\int_{R}(\exp(-x^{2})^{(n)})^{2}\de x \\
                                     &< +\infty
.\end{align*} Ainsi $\psi$ est bien $L^2$ par rapport à $x$.
\end{document}
