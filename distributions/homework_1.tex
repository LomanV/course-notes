\documentclass[12pt]{article}
\usepackage[top=1in, bottom=1in, left=1in, right=1in]{geometry}

\usepackage[onehalfspacing]{setspace}

\usepackage{amsmath, amssymb, amsthm}
\usepackage{enumerate, enumitem}
\usepackage{fancyhdr, graphicx, proof, comment, multicol}
\usepackage[none]{hyphenat}
\usepackage{dirtytalk}
\binoppenalty=\maxdimen
\relpenalty=\maxdimen

\usepackage{microtype}
\usepackage{mathpazo}
\usepackage{mdframed}
\usepackage{parskip}
\linespread{1.1}
\usepackage{graphicx}
\usepackage{subfig}

\usepackage{mathrsfs}
\usepackage{amsfonts}
\usepackage{amsmath}
\usepackage{amssymb}

\usepackage{mathtools}
\newcommand{\defeq}{\vcentcolon=}
\newcommand{\eqdef}{=\vcentcolon}

\newenvironment{statement}[1]
{\begin{mdframed}[linewidth=0.6pt]
        \textsc{Statement #1:}

}

\newenvironment{ex}[1]
{\begin{mdframed}[linewidth=0.6pt]
        \textsc{Exercice #1:}

}
    {\end{mdframed}}

\newcommand{\R}{\mathbf{R}}
\newcommand{\C}{\mathbf{C}}
\newcommand{\Z}{\mathbf{Z}}
\newcommand{\N}{\mathbf{N}}
\newcommand{\Q}{\mathbf{Q}}

\newcommand{\de}{\mathrm{d}}

\begin{document}
        \noindent
\textbf{Distributions et EDP} \hfill \textbf{Vezin Lomàn}\\
\normalsize MAT432 \hfill Date de rendu: 05/01/2021\\

\begin{center}
\textbf{Devoir maison}
\end{center}

\textbf{I.} \textit{1.} On détermine le support de $T_{n}$ définie comme dans l'énoncé. On a pour $n \in \N^{*}$, $h \in \R^{*}$ et $\phi \in \mathcal{C}_{c}^{\infty}(\R)$
\begin{align*}
        \langle T_{n}, \phi \rangle &= \langle \delta_{0} + \sum_{k=1}^{n} \frac{h^{k}}{k!}\delta^{(k)}_{0}, \phi \rangle \\
                                    &= \langle \delta_{0}, \phi \rangle + \sum_{k=1}^{n} \frac{h^{k}}{k!}\langle \delta^{(k)}_{0}, \phi \rangle \\
                                    &= \phi(0) + \sum_{k=1}^{n} \frac{h^{k}}{k!}\phi^{(k)}(0) 
.\end{align*}

Le support de la dirac $\delta_{0}$ et de ses dérivées $\delta^{(k)}_{0}, \; k \le n$ est le singleton $\{0\}$. On voit alors facilement que la restriction de $T_{n}$ à $\R \setminus \{0\}$ est nulle et donc par définition du support d'une distribution \[
        \boxed{\mathrm{supp}(T_{n}) = \{0\}.} 
\] 

\bigskip

\textit{2.} Montrons que $\mathrm{supp}(S) \subset K$.
Supposons par l'absurde que ce ne soit pas le cas, c'est à dire $\mathrm{supp}(S)\setminus K \neq \emptyset$. Soit $\Omega$ un voisinage ouvert de $\mathrm{supp}(S)$. $\Omega \setminus K \neq \emptyset$ et par définition du support $\langle S_{n}, \phi \rangle = 0$ pour tout $\phi \in \mathcal{C}_{c}^{\infty}(\R)$ à support dans $\mathrm{supp}(S)\setminus K$ et tout $n \in \N^{*}$. Aussi par définition du support $S$ n'est pas identiquement nulle sur $\Omega \setminus K$ ce qui nous donne l'existence de $\psi \in \mathcal{C}_{c}^{\infty}(\R)$ à support dans $\mathrm{supp}(S)\setminus K$ telle que \[
        \langle S, \psi \rangle \neq 0
.\] Mais alors pour cette fonction $\psi$  \[
0 = \langle S_{n}, \psi \rangle \not\to \langle S, \psi \rangle \neq 0 
,\] ce qui contredit $S_{n} \to_{n} S$ au sens des distributions. On a donc bien \[
        \boxed{\mathrm{supp}(S) \subset K.}
\]  
        
\bigskip

\textit{3.} Par définition de $T_{n}$ on obtient \[
        \star = \langle T_{n}, \phi_{|_{\R}} \rangle = \phi(0) + \sum_{k=1}^{n} \frac{h^{k}}{k!}(-1)^{k}\phi^{(k)}(0)
.\] 

En se souvenant que les fonctions holomorphes sur un domaine sont analytiques sur ce domaine, on en déduit en reconnaissant l'expression du développement de Taylor à l'ordre $n$ de $\phi(-h)$ et comme $|h| \le R$ que \[
        \star = \phi(0-h) + o(h^{n}) = \phi(-h) + o(h^{n})
.\] 

En particulier l'analycité de $\phi$ garantit  \[
        \phi(-h) = \phi(0) + \sum_{k=1}^{\infty} \frac{(-h)^{k}}{k!}\phi^{(k)}(0)
        .\] La suite $\langle T_{n}, \phi \rangle$ converge donc et on trouve naturellement \[
\boxed{\lim_{n\to \infty}\langle T_{n}, \phi \rangle = \phi(-h).}
\]  

\medskip

\textit{4.} Dans la question \textit{1.} nous avons montré que le support de $T_{n}$ est $\mathrm{supp}(T_{n}) = \{0\}$ pour tout $n \in \N$. On en déduit que la suite de distributions $(T_{n})_{n\in \N}$ ne converge pas au sens des distributions puisque sinon par la question \textit{2.} le support de la limite devrait appartenir au singleton $\{0\}$, ce qui n'est pas le cas comme $h \neq 0$. 

\newpage

\textbf{II.} \textit{1.} Posons pour plus de simplicité $H_{n} \defeq \sum_{k=1}^{n} \frac{1}{k}$ la $n-$ème somme partielle de la série harmonique. On montre que la suite $H_{n} - \log(n)$ converge et que sa limite est positive. Fixons $n \in \N_{*}$.

Dans un premier temps remarquons que sur $\R_{+}^{*}$ la fonction inverse $t \longmapsto \frac{1}{t}$ est décroissante. Ainsi pour $1 \le k \le n-1$ et $t \in [k, k+1]$ on obtient \[
\frac{1}{k+1} \le \frac{1}{t} \le \frac{1}{k}
.\] 
On intègre par rapport à $t$ sur les intervalles $[k, k+1]$ pour obtenir par croissance de l'intégrale et comme tous les termes sont positifs
\[
\frac{1}{k+1} \le \int_{k}^{k+1} \frac{\de t}{t} \le \frac{1}{k}
.\] 

On en déduit d'une part \[
        \frac{1}{k} - \int_{k}^{k+1} \frac{\de t}{t} \ge 0 
,\] et d'autre part que \[
\frac{1}{k} - \int_{k}^{k+1}\frac{\de t}{t} \le \frac{1}{k} - \frac{1}{k+1}
.\]  

Ces deux inégalités nous donnent donc l'encadrement \[
0 \le \frac{1}{k} - \int_{k}^{k+1}\frac{\de t}{t} \le \frac{1}{k} - \frac{1}{k+1}
.\] 

Comme la série de terme général $\frac{1}{k}-\frac{1}{k+1}$ converge et que tous les termes sont positifs par comparaison la série de terme général 
\[
        \frac{1}{k} - \int_{k}^{k+1} \frac{\de t}{t} = \frac{1}{k} - (\log(k+1) - \log(k))
\] converge également et la limite et positive. Finalement sommons sur $1 \le k \le n-1$ pour trouver la $n$-ème somme partielle de la série en question 
\begin{align*}
        H_{n-1} - \sum_{k=1}^{n-1} \log(k+1) - \log(k) = H_{n-1} - \log(n) 
,\end{align*}
en reconnaissant une somme télescopique. En particulier grâce aux encadrements précédents on trouve que la limite est inférieure à 1 comme \[
        0 \le H_{n} - \log(n) \le 1
.\]
La suite de terme général \[
        H_{n} - \log(n) = H_{n-1} - \log(n) + \frac{1}{n}
\] converge également, vers la même limite, par convergence de la suite de terme général $\frac{1}{n}$, vers 0.

\bigskip

\textit{2.} On applique le point \textit{c.} du théorème \textbf{4.2} du cours, selon lequel si $T$ est une distribution à support compact dans un ouvert $\Omega \subset \R^{N}$ et si $(\phi_{n})_{n}$ est une suite de fonctions $\mathcal{C}^{\infty}(\Omega)$ telle que pour tout $\alpha \in \N^{N}$ on a \[
\partial^{\alpha}\phi_{n} \to_{n} \partial^{\alpha}\phi
\] uniformément sur les compacts de $\Omega$ alors  \[
\langle T, \phi_{n} \rangle \to_{n} \langle T, \phi \rangle
.\] 

Dans notre cas, par hypothèse le support de $T$ est inclus dans $V \subset \Omega$ ouvert donc on peut regarder $T$ comme une distribution à support compact dans $V$. De plus $\mathcal{C}^{\infty}(\Omega) \subset \mathcal{C}^{\infty}(V)$ comme $V \subset \Omega$ et on a  \[
\partial^{\alpha}\phi_{n} \to_{n} \partial^{\alpha}\phi \quad \text{uniformément sur } V
,\] donc en particulier sur tout compact de $V$. 

Avec $V$ qui joue le rôle de $\Omega$ dans l'énoncé ci dessus on obtient pour toute suite $(\phi_{n})_{n}$ de $\mathcal{C}^{\infty}(\Omega)$, donc de $\mathcal{C}^{\infty}(V)$ comme ci dessus \[
        \langle T, \phi_{n} \rangle \to_{n} \langle T, \phi \rangle
.\]  

\bigskip

\textit{3.} Soit $m \in \N^{*}$ et soit $\phi \in \mathcal{C}^{\infty}_{c}(\R)$, la formule du développement de Taylor d'ordre 1 en 0 avec $h = \frac{1}{k}$ nous donne
\begin{align*}
        \langle T_{m}, \phi \rangle &= \sum_{k=1}^{m} \phi(\frac{1}{k}) - m\phi(0) - \log(m)\phi'(0) \\
                                    &= \sum_{k=1}^{m}(\phi(0) + \frac{1}{k}\phi'(0) + o(\frac{1}{k^{2}})) - m\phi(0) - \log(m)\phi'(0) \\
                                    &= (\sum_{k=1}^{m}\frac{1}{k} - \log(m))\phi'(0) + \sum_{k=1}^{m} o(\frac{1}{k^{2}}) 
.\end{align*}

Par le point \textit{1.} nous savons que la limite suivante existe \[
\sum_{k=1}^{m}\frac{1}{k} - \log(m) \to_{m} \gamma
.\] De plus par convergence de la série des inverses au carré il existe une constante réelle finie $C$ telle que \[
\sum_{k=1}^{\infty} o(\frac{1}{k^{2}}) \le C 
.\]

Ainsi la limite des $T_{m}$ est bien une distribution que l'on peut nommer $T$. En exprimant le terme de reste par une intégrale on trouve 
\[
        \boxed{\langle T, \phi \rangle = \gamma\phi'(0) + \sum_{k=1}^{\infty} \int_{0}^{\frac{1}{k}}\frac{(\frac{1}{k}-t)^{2}}{2}\phi^{(2)}(t)\de t.}
\] 
 telle que $\langle T_{m}, \phi \rangle \to_{m} \langle T, \phi \rangle$. Par choix arbitraire de $\phi \in \mathcal{C}^{\infty}_{c}(\R)$ on a \[
T_{m} \to_{m} T
.\] 

\bigskip

\textit{4.} $T$ a pour support l'ensemble $\{\frac{1}{n} \;|\; n \in \N^{*}\} \cup \{0\} $. Pour $m \in \N^{*}$ le support de $T_{m}$ est quant à lui l'ensemble $\{\frac{1}{k} \;|\; 1\le k \le m\} \cup \{0\}$. 
\medskip

\textit{5.} L'ordre de $T$ vaut 1. Cela est donné par la majoration de l'intégrale \[
        \boxed{\sum_{k=1}^{\infty} \int_{0}^{\frac{1}{k}}\frac{(\frac{1}{k}-t)^{2}}{2}\phi^{(2)}(t)\de t \le \|\phi'\|\sum_{n=1}^{\infty} \frac{1}{k^{2}}.} 
\]  De plus par le point \textit{3.} on obtient pour $m \in \N^{*}$ et $\phi \in \mathcal{C}_{c}^{\infty}(\R)$ quelconques \[
|\langle T_{m}, \phi \rangle| \le (\sum_{k=1}^{m} \frac{1}{k} - \log(m))|\phi'(0)| + \sum_{k=1}^{m} o(\frac{1}{k^{2}})
,\] en particulier les deux sommes sont bornées puisque convergentes et il existe une constante réelle $C$ telle que \[
\boxed{|\langle T_{m}, \phi \rangle| \le C|\phi'(0)|,}
\] ce qui montre que $T_{m}$ est également d'ordre 1.\footnote{Techniquement nous avons montré que $T$ et $T_{m}$ sont d'ordre au plus 1, pour montrer qu'elles ne peuvent pas être d'ordre 0 on peut exhiber des fonctions test similaires à celles que nous avons utilisées pour établir que la $k-$ème dérivé de la Dirac est d'ordre $k$.} 

\bigskip

\textit{6.} Soit $\phi \in \mathcal{C}^{\infty}(\R)$ nulle sur le support de $T$. Soit $k \in \N^{*}$ quelconque, on écrit le développement de Taylor à l'ordre 1 en $\frac{1}{k}$
\[
        0 = \phi(0) - \phi(\frac{1}{k}) = \frac{1}{k}\phi'(0) + o(\frac{1}{k})
.\] En multipliant par $k$ des deux côtés on obtient donc  \[
\phi'(0) = o(\frac{1}{k})\cdot k
.\] Le résultat étant valable pour tout $k \in \N^{*}$ on peut passer à la limite $k \to \infty$ et on obtient  \[
\phi'(0) = 0
.\] Pour $m \in \N^{*}$ quelconque, $\phi$ s'annule sur $\{\frac{1}{k} \;|\; k \le m\} \cup \{0\}$ et donc par définition de $T_{m}$ on a alors $\langle T_{m}, \phi \rangle = 0$. Puis par convergence au sens des distributions $T_{m} \to_{m} T$ on en conclut \[
\boxed{\langle T, \phi \rangle = 0.}
\]  

\medskip

\textit{7.} Soit $\varepsilon > 0$ et soit $n \in \N$. Pour $x \in \R$ quelconque on a \[
        |\phi_{n}(x) - 0| = \phi_{n}(x) \le \frac{1}{\sqrt{n}} \\
.\] Ainsi pour tout $n \ge \frac{1}{\varepsilon^{2}}$ on obtient \[
\phi_{n}(x) \le \varepsilon
,\] ce qui montre la convergence uniforme $\phi_{n} \to_{n} 0$. 

Par définition de $\phi_{n}$, pour $k \in \N$ quelconque
\[
        \phi^{(k)}(x) = 0, \quad \forall x \in \R\setminus[\frac{1}{n+1}, \frac{1}{n}]
.\] Ainsi les seuls points du support de $T$ où $\phi_{n}^{(k)}$  n'est pas forcément nulle sont $\{\frac{1}{n+1}, \frac{1}{n}\}$.
Comme $\phi$ est $\mathcal{C}^{\infty}(\R)$ sa dérivée à l'ordre $k$ est en particulier continue et donc \[
        \phi^{(k)}(\frac{1}{n}) = \lim_{x\to \frac{1}{n}^{+}} \phi^{(k)}(x) = 0
\] où la limite est prise par la droite et de même \[
        \phi^{(k)}(\frac{1}{n+1}) = \lim_{x\to \frac{1}{n+1}^{-}} \phi^{(k)}(x) = 0
\] en prenant la limite par la gauche. 

$\phi_{n}^{(k)}$ est donc nulle sur le support de $T$ et converge en particulier uniformément vers 0 sur ce dernier par choix arbitraire de $n \in \N$.
\medskip

\textit{8.} Soit $m \in \N^{*}$ et soit $n \ge 1$, on calcule $\langle T_{m}, \phi_{n}\rangle$.
\begin{align*}
        \langle T_{m}, \phi_{n} \rangle &= \sum_{k=1}^{m} \phi_{n}(\frac{1}{n}) - m\phi_{n}(0) - \log(m)\phi'_{n}(0) \\
                                        &= \sum_{k=1}^{m} \phi_{n}(\frac{1}{k}) \quad \text{comme par définition on a } \phi_{n}(0), \phi'_{n}(0) = 0 \\
                                        &= \sum_{k=1}^{n} \frac{1}{\sqrt{n}} \\
                                        &= \frac{n}{\sqrt{n}}
.\end{align*}

On en déduit par passage à la limite $m \to \infty$ \[
        \boxed{\langle T, \phi_{n} \rangle = \frac{n}{\sqrt{n}}.} 
\] 
\medskip

\textit{9.} On remarque que $\phi_{n} \to_{n} 0$ uniformément sur $\R$ et les dérivées $k-$ème tendent toutes uniformément vers 0 sur le support de $T$ mais contrairement au point \textit{2.} on a 
 \[
         \boxed{\langle T, \phi_{n} \rangle \not\to_{n} \langle T, 0 \rangle}
\] puisque le premier terme diverge alors que le deuxième vaut 0. 

L'hypothèse que les dérivées successives convergent uniformément \emph{au voisinage} du support et non seulement sur ce dernier est donc primordiale.
\end{document}
