\documentclass[12pt]{article}
\usepackage[top=1in, bottom=1in, left=1in, right=1in]{geometry}

\usepackage[onehalfspacing]{setspace}

\usepackage{amsmath, amssymb, amsthm}
\usepackage{enumerate, enumitem}
\usepackage{fancyhdr, graphicx, proof, comment, multicol}
\usepackage[none]{hyphenat}
\usepackage{dirtytalk}
\binoppenalty=\maxdimen
\relpenalty=\maxdimen

\usepackage{microtype}
\usepackage{mathpazo}
\usepackage{mdframed}
\usepackage{parskip}
\linespread{1.1}
\usepackage{graphicx}
\usepackage{subfig}

\usepackage{mathrsfs}
\usepackage{amsfonts}
\usepackage{amsmath}
\usepackage{amssymb}

\usepackage{mathtools}
\newcommand{\defeq}{\vcentcolon=}
\newcommand{\eqdef}{=\vcentcolon}

\newenvironment{statement}[1]
{\begin{mdframed}[linewidth=0.6pt]
        \textsc{Statement #1:}

}
    {\end{mdframed}}

\newcommand{\R}{\mathbf{R}}
\newcommand{\C}{\mathbf{C}}
\newcommand{\Z}{\mathbf{Z}}
\newcommand{\N}{\mathbf{N}}
\newcommand{\Q}{\mathbf{Q}}

\begin{document}
\noindent \textbf{Vezin Lomàn} \hfill \textbf{Analyse IV}\\
\normalsize Prof. D. Strütt  \hfill Due Date: 26/03/2020\

\begin{center}
\textbf{Homework 5}
\end{center}

\begin{statement}{1}
	Soient $f,g \ge 0$ des fonctions mesurables, alors \[
		\int (f+g) = \int f + \int g
	.\] 
\end{statement}
\begin{proof}
	On commence par montrer le résultat pour deux fonctions simples positives $\varphi, \psi$.
	 \[
	\varphi = \sum_{i=1}^{n} a_{i}\chi_{A_{i}} \qquad \psi = \sum_{j=1}^{m} b_j \chi_{B_{j}}
.\] Ainsi comme les $A_{i}$ et les $B_{j}$ forment une partition de $\mathbf{R}$ on obtient que 
\[
	\varphi + \psi = \sum_{1\le i\le n \; , \; 1 \le j \le m} (a_{i} + b_{j})\chi_{A_{i} \cap B_{j}}
.\] 
Donc
\begin{align*}
	\int (\varphi + \psi) &= \sum_{1\le i\le n \; , \; 1 \le j \le m} (a_{i} + b_{j}) mes(A_i \cap B_j) \\
			      &= \sum_{1\le i\le n \; , \; 1 \le j \le m} a_{i} mes(A_i \cap B_j) + \sum_{1\le i\le n \; , \; 1 \le j \le m} b_j mes(A_i \cap B_j) \\
			      &= \sum_{i=1}^{n} a_i mes(A_i) + \sum_{j=1}^{m} b_j mes(B_j) \\
			      &= \int \varphi + \int \psi
\end{align*}
puisque les $A_{i}, B_{j}$ forment une partition de $\mathbf{R}$ et on le résultat.
Maintenant,
\begin{align*}
	\int f + \int g &= \sup \{\int \varphi \;|\; 0 \le \varphi \le f \} + \sup \{ \int \psi \;|\; 0 \le \psi \le g\} \\
		&= \sup \{\int \varphi + \int \psi \;|\; 0 \le \varphi \le f \; 0 \le \psi \le g \} \\
		&= \sup \{\int (\varphi + \psi) \;|\; 0 \le \varphi + \psi \le f + g\} \\
		&= \int (f+g)
.\end{align*}
Ce qui conclut la preuve.
\end{proof}

\begin{statement}{2}
	Pour $f_{n}\ge 0$ une suite de fonctions mesurables,
	\[
	\int \sum_{n=1}^{\infty} f_{n} = \sum_{n=1}^{\infty} \int f_n
	.\] 	
\end{statement}
\begin{proof}
	On montre dans un premier temps par récurrence sur $m > 0$ que \[
	\int \sum_{n=1}^{m} f_n = \sum_{n=1}^{m} \int f_n
	.\] 
	Par le premier point nous savons déjà que le résultat est valide pour $m = 2$.
	Supposons que le résultat soit vérifié pour $m \ge 2$, alors 
	 \begin{align*}
		 \sum_{n=1}^{m+1} \int f_{n} &= \sum_{n=1}^{m} \int f_{n} + \int f_{m+1} \\
					     &= \int \sum_{n=1}^{m} f_{n} + \int f_{m+1} \quad \text{par hypothèse de récurrence} \\
					     &= \int (\sum_{n=1}^{m} f_{n} + f_{m+1}) \quad \text{par l'additivité montrée au premier point} \\
					     &= \int \sum_{n=1}^{m+1} f_{n} \quad \text{et le résultat est démontré.}
	\end{align*} 
	Remarquons à présent que la suite $(\sum_{n=1}^{m} f_n)_{m=1}^{\infty}$ est une suite croissante de fonctions mesurables puisque les $f_{n}$ sont des fonctions positives et mesurables. Ainsi par le théorème de la convergence monotone \[
		\int \sum_{n=1}^{\infty} f_{n} = \int \lim_{m \rightarrow \infty} \sum_{n=1}^{m} f_{n} = \lim_{m \rightarrow \infty} \int \sum_{n=1}^{m} f_{n} = \lim_{n \rightarrow \infty} \sum_{n=1}^{m} \int f_{n} = \sum_{n=1}^{\infty} \int f_{n} 
.\]  Ce qui conclut la preuve.
\end{proof}
\end{document}
