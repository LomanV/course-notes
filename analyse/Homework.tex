\documentclass[12pt]{article}
\usepackage[top=1in, bottom=1in, left=1in, right=1in]{geometry}

\usepackage[onehalfspacing]{setspace}

\usepackage{amsmath, amssymb, amsthm}
\usepackage{enumerate, enumitem}
\usepackage{fancyhdr, graphicx, proof, comment, multicol}
\usepackage[none]{hyphenat}
\usepackage{dirtytalk}
\binoppenalty=\maxdimen
\relpenalty=\maxdimen

\usepackage{microtype}
\usepackage{mathpazo}
\usepackage{mdframed}
\usepackage{parskip}
\linespread{1.1}
\usepackage{graphicx}
\usepackage{subfig}

\usepackage{mathrsfs}
\usepackage{amsfonts}
\usepackage{amsmath}
\usepackage{amssymb}

\usepackage{mathtools}
\newcommand{\defeq}{\vcentcolon=}
\newcommand{\eqdef}{=\vcentcolon}

\newenvironment{statement}[1]
{\begin{mdframed}[linewidth=0.6pt]
        \textsc{Statement #1:}

}
    {\end{mdframed}}

\newcommand{\R}{\mathbf{R}}
\newcommand{\C}{\mathbf{C}}
\newcommand{\Z}{\mathbf{Z}}
\newcommand{\N}{\mathbf{N}}
\newcommand{\Q}{\mathbf{Q}}

\begin{document}
	\noindent
\textbf{Analyse IV} \hfill \textbf{Vezin Lomàn}\\
\normalsize Prof. D. Strut \hfill Due Date: 02/04/2020\\

\begin{center}
\textbf{Homework}
\end{center}
	\begin{statement}{1}
		Soit $f \in L^p$ et $1 \le p < \infty$, alors \[
			\lim_{\varepsilon \rightarrow 0} \int_{\mathbf{R}}|f(x+\varepsilon) - f(x)|^p dx = 0
		.\] 	
	\end{statement}
	\begin{proof}
		On commence par montrer le résultat pour une fonction $g \in \mathcal{C}_{0}^{\infty}(\mathbf{R})$. \\
		Soit $\gamma > 0$ quelconque et soit $[a,b] \subset \mathbf{R}$ un intervalle contenant le support compact de $g$. Puisque $g$ est continue sur ce compact elle est en particulier uniformément continue et il existe $\delta > 0$ tel que pour $\varepsilon < \delta$ on a $|g(x+\varepsilon)-g(x)| \le {(\frac{\gamma}{3\cdot(b-a)})}^{\frac{1}{p}}$. Ainsi on obtient \[
			\int_{\mathbf{R}}|g(x+\varepsilon)-g(x)|^pdx = \int_a^b|g(x+\varepsilon)-g(x)|^pdx \le \frac{\gamma}{3}
		.\] 
		Puisque $\gamma > 0$ est arbitraire on a \[
			\lim_{\varepsilon \rightarrow 0} \int_{\mathbf{R}}|g(x+\varepsilon)-g(x)|dx = 0
		.\] 
		Soit maintenant $\gamma > 0$ quelconque et $f \in L^{p}$, alors par le théorème d'approximation par fonction lisses il existe $g \in \mathcal{C}_{0}^{\infty}(\mathbf{R})$ telle que $\|f-g\|_{L^{p}} \le \frac{\gamma}{3}$.
		Si on pose les fonctions auxiliaires $\hat{f}(x) = f(x+\varepsilon)$ et $\hat{g}(x) = g(x+\varepsilon)$ pour tout $x \in \mathbf{R}$ on remarque par le point précédent que \[
			\|\hat{f}- \hat{g}\|_{L^{p}} = \|f-g\|_{L^{p}} \le \frac{\gamma}{3} \quad \text{et} \quad \|\hat{g} - g\|_{L^{p}} \le \frac{\gamma}{3}
		.\] 
		Ainsi par la feinte du loup et par inégalité triangulaire on obtient 
		\begin{align*}
			\|\hat{f}-f\|_{L^{p}} &= \|\hat{f}-\hat{g}-f+g-g+\hat{g}\|_{L^{p}} \\
					      &\le \|\hat{f}-\hat{g}\|_{L^{p}} + \|f-g\|_{L^{p}} + \|\hat{g}-g\|_{L^{p}} \\
					      &\le \gamma
		.\end{align*}
		Ainsi lorsque $\varepsilon$ tend vers 0 on a bien $\|\hat{f}-f\|_{L^{p}} \longrightarrow 0$ puisque $\gamma$ est arbitraire. En d'autres termes  \[
		\lim_{\varepsilon \rightarrow 0} (\int_{\mathbf{R}}{|f(x+\varepsilon)-f(x)|^pdx})^{\frac{1}{p}} = 0
		.\] 
		Soit donc le résultat voulu \[
		\lim_{\varepsilon \rightarrow 0} \int_{\mathbf{R}}|f(x+\varepsilon)-f(x)|^pdx = 0
		.\] 
	\end{proof}
\end{document}
