\documentclass {article}
\usepackage[francais]{babel}
\usepackage[top=3cm, bottom=3cm, left=2.5cm, right=2.5cm]{geometry}
\usepackage{amsmath}
\usepackage{textcomp}
\title{SHS - Climat 2019}
%\author{Lomàn Vezin}
\date{\vspace{-5ex}}

\newcommand{\V}[0]{\vspace{1\baselineskip}}

\begin{document}
\maketitle
\bigskip

\section{Définitions}
\paragraph{}
\textit{Climat :} Description statistique des grandeurs pertinentes, état du système climatique.
\textbf{Différent de la météo} qui décrit le temps qu'il fait, les \emph{échelles temporelles} sont différentes. La météo dépend des conditions initiales, le climat des conditions limites.
\vspace{1\baselineskip}
\\
\textit{Système climatique :} 
\begin{itemize}
\item Atmosphère, hydrosphère, cryosphère, lithosphère, biosphère
\item Influencé par la \emph{dynamique propre} et les \emph{forçages externes} (soleil, etc.)
\end{itemize}
\vspace{1\baselineskip}
\textit{Atmosphère :} Enveloppe gazeuse qui entourre la Terre (78\% N$_2$, 21\% O$_2$, 0.04\% CO$_2$)
\V
\\
\textit{Hydrosphère :} Partie liquide de l'eau terrestre, océans, lacs, rivières, eaux souterraines.
\V
\\
\textit{Cryosphère :} Toutes les régions qui contiennent de l'eau sous forme solide (calotte polaire, banquise, Pergélisol, etc.)
\V
\\
\textit{Lithosphère : } Couche de la croute terrestre et partie supérieure du manteau, se comporte de manière élastique. Responsable majeure des variations de l'albedo (pourcentage du rayonnement solaire réfléchi à la surface), qui influence le climat à la surface ainsi que le taux d'humidité.
\V
\\
\textit{Biosphère :} Somme de tous les écosystèmes sur Terre (atmosphériques et océaniques) et matière organique morte. Incidence sur l'albedo et la rugosité de la surface.
\V
\\
\textit{Anthroposphère :} Aujourd'hui les effets de l'Homme sur son environnement son comparables aux forces telluriques, on parle d'ère \emph{anthropocène}.
\V
\\
\textit{Thermoaline :} Ensemble des courants océaniques dus aux gradients de densité dans les océans

\bigskip

\section{Modèles climatiques et projections}

\paragraph{}
\textbf{Définition:} \textit{modèle climatique} :
\\ Ensemble d'équations différentilles aux dérivées partielles décrivant le système climatique
\begin{itemize}
\item Résolu numériquement pas analytiquement
\item Processus parfois paramétrés mais pas directement représentés
\end{itemize}

\paragraph{}
\textbf{Différents modèles :} 

\paragraph{}
\textit{Modèles globaux} :
\begin{itemize}
\item Modèles récents comme \textbf{AOGCM ESM}, qui incluent des représentations de divers cycles biogéochimiques (carbone, souffre)
\item Les plus avancés aujourd'hui
\item Résolution typique : \textbf{200km}
\end{itemize}

\paragraph{}
\emph{Modèles régionaux} :
\begin{itemize}
\item Aire limitée
\item \textbf{RCM LAM}
\item Représentation semblable aux modèles globaux (mais pas interaction banquise, océan, etc.)
\item Désagrégation des simulations
\item résolution typique : 50km
\item Plus fin mais pour une région donnée
\item \textbf{Pas un modèle global !}
\end{itemize}

\subsection{Incertitudes}

\paragraph{}
\begin{itemize}
\item Incertitudes dans les \emph{modèles}, imperfection des modèles, simplification
\item Incertitudes dans le \emph{forçage}, émissions, volcanisme, etc.
\item Incertitude de la \emph{variabilité naturelle} du système climatique
\\
\item Fourni une enveloppe statistique, incertitudes plus importantes au plus on s'éloigne
\item On constate une réduction des incertitudes avec l'amélioration des modèles
\item Temps court forte influence des conditions initiales, variabilité naturelle domine
\item Après 20 ans, les erreurs dans les modèles genèrent le plus d'erreur dans les prévisions climatiques
\item Après 40 ans c'est le forçage (encore une fois relatif à la région, et la variable considérée)
\end{itemize}
\bigskip

\section{Changements climatiques actuels et futurs}

Le \textbf{GCOS} a défini \textbf{50} variables climatiques essentielles (atmosphère, océan, surface terrestre, etc.)
\\
\begin{itemize}
\item Observation à la \emph{surface} : station météorologique e.g. MCH à Aigle, réseau Suisse MCH 
\item Observation de \emph{l'atmosphère} : radiosondage, ballon munit d'une sonde qui effectue des mesures pendant son ascension
\item 30 min explose à 20km d'altitude, mesure toute la troposphère \\
\\
\item Réseau mondial, densité de stations inégale, océans moins observés que la surface
\item Satellites pour observer la Terre depuis l'espace, ils offrent une surveillance globale, océan et surface, c'est pourquoi ils sont de plus en plus utilisées
\end{itemize}

\subsection{Changements climatiques actuels}

\paragraph{}
\begin{itemize}
\item Augmentation en surface et au niveau de la troposphère
\item Mesures, pas modèles
\item Réchauffement \textbf{non uniforme} (spatialement et saisionièrement)
\item Réchauffement moyen de 0.87\textdegree C sur 2006-2015 / 1850-1900
\item Les continents se réchauffent plus que les océans, surtout l'hiver dans l'hémisphère Nord
\item 1951-2010 principalement du aux activités humaines (forçage naturel négligeable)
\item Augmentation de la concentration des gaz à effet de serre
\\
\item Augmentation des précipitations dans l'hémisphère Nord
\item Plus d'incertitude que pour la température (observation et mesures plus difficiles)
\item Tendances peu fiables dans les autres régions
\\
\item Changements dans les extremes : plus de jours chauds, sécheresse, canicule, précipitations intenses, crues, etc.
\end{itemize}

\subsection{Changements climatiques futurs}

\begin{itemize}
\item Besoin de scénarios pour les forçages, notamment humains
\item SRES / RCP (on fixe seulement la concentration)
\end{itemize}
\bigskip

\section{Eléments de Biogéographie}

\textbf{Définition :} \textit{Ecozone} ou \textit{Realm} :
\\ Plus large division biogéographique sur Terre. Délimite un vaste espace géographique dans lequel les organismes ont évolué de manière isolée.
\paragraph{}
\begin{itemize}
\item similarité floristique et faunistique
\item intègre des facteurs historiques (isolation géographique passée)
\item "barrière" naturelle (océan, plaques techtoniques)
\item très large zone
\item 6/11 régions biogéographiques
\\
\item Les biomes (différence écozone) : ensemble d'écosystèmes majeurs caractéristiques d'une aire biogéographique nommé à partir de la végétation
\item adaptation des espèces présominantes
\item 9 à connaitre
\end{itemize}

\paragraph{}
\textbf{Définition :} \textit{Biogéographie} :
\\ Etude de la distribution des espèces végétales et animales dans les écosystèmes

\paragraph{Chronologie}
\begin{itemize}
\item 1761 Buffon constate différences entre mammifères ancien et nouveau monde
\item 1816 von Humboldt étend la loi de Buffon aux oiseaux, reptiles et plantes à fleurs
\item 1820 de candolle formalise le premier système mondial de régions biogéographiques
\item 1858 Scalter et Engler définissent 6 grandes régions biogéographiques
\item 1820, 1876 Wallace confirme ces travaux
\item 2012 Holt et al. améliorent la précision des zones avec à l'appui l'outil génétique

\item Température / précipitation / saisonnalité influencent les biomes (e.g. mousson)
\item Topographie, le sol, feu, herbivorie sont des facteurs secondaires
\end{itemize}

\paragraph{Biomes}
\textbf{9 biomes} selon Robert Whittaker / 9 zones climatiques de Heinrich Walter

\paragraph{}
\emph{3 biomes tropicaux} : $>$ 20\textdegree C\\
\begin{enumerate}
\item Forêt tropicale humide 40-75\% biodiversité 13\% surface humide et chaud toute l'année
\item Savanes été humide hiver sec
\item Désert subtropical sécheresse et chaleur toute l'année 20\% biomes planète
\end{enumerate}

\paragraph{}
\emph{4 biomes tempérés} : 5-20\textdegree C\\
\begin{enumerate}
\item Végétation méditérrannéne 3\% biomes mais la plus menacée par le réchauffement climatique
\item Forêt tempérée humide températures fraiches, arbres les plus hauts de la planète
\item Forêt tempérée caducifoliée forte saisonnalité, humide, précipitations abondantes
\item Prairie tempérée / desert froid fort contraste de température, climat sec
\end{enumerate}

\paragraph{}
\emph{2 lattitudes polaires et boréales} : $<$ 5\textdegree C\\
\begin{enumerate}
\item Taïga, forêt boréale : 17\% de la surface terrestre
\item Toundra (Arctique et Alpine) fort contraste thermique, hivers longs et froids (6-10 mois et températures négatives)
\end{enumerate}
\bigskip

\section{Vulnérabilité des biomes face aux changements globaux}

\subsection{Dégradation des biomes}

On observe une fragmentation importante des biomes liée aux activitées humaines, principalement l'agriculture extensive

\paragraph{Services biologiques}
\textbf{Définition :} \textit{services écologiques} :
\\ Bénéfices que les humains retirent des écosystèmes, ils se séparent en \textbf{3} catégories :
\\
\begin{itemize}
\item \textit{Approvisionnement} : produits tangibles tirés des écosystèmes
\item \textit{Régulation} : avantages intengibles liés au bon fonctionnement des écosystèmes
\item \textit{Socioculturels} : apports non matériels, améliore la qualité de vie humaine
\end{itemize}

\paragraph{}
\textbf{MEA} ($>$ 1300 experts et 50 pays) rapporte que \textbf{60\%} des services écologiques de la planète sont dégradés.

\paragraph{Services écosystémiques}
En EOC et en Asie centrale on observe une baisse des contributions de la nature aux populations ainsi qu'une forte diminution des services de régulation et une tendance positive des services d'approvisionnements liés à l'agriculture intensive.
Les services écosystemiques de régulation sont largement affectés par les changements globaux.

\subsection{Rôle de la biodiversité}

\paragraph{}
\textbf{Définition :} \textit{biodiversité} : 
\\ Variabilité des organismes vivants, diversité des espèces, finir...

\paragraph{}
Il existe \textbf{3} niveaux de biodiversité :
\begin{enumerate}
\item \textit{Ecosystème :} diversité d'écosystèmes dans une surface donnée.
\item \textit{Inter espèce :} Diversité d'espèces au sein d'un écosystème.
\item \textit{Intra espèce :} Variabilité génétique au sein d'une espèces, différents allèles
\end{enumerate}

\paragraph{}
\emph{La biodiversitée est \textbf{hautement} affectée par les différentes composantes du changement global}.

La richesse spécifique des biomes est de plus \emph{inégalement} répartie : moins riche Tundra, plus riche forêts tropicales humides

\paragraph{Erosion de la biodiversité}
On constate sur plus de 50\% des terres une perte supérieure à 20\% de la biodiversité, notamment des extinctions massives, on parle de "6\up{e} crise biologique" induite par l'Homme.
Ainsi cette perte constitue un \emph{enjeu Majeur du 21\up{e} siecle}.
De plus les tendances se poursuivent et par endroits s'accentuent. 
Ainsi 28\% des especes menacées d'extinction en Europe et Asie centrale.

\paragraph{5 facteurs anthropiques}
\begin{enumerate}
\item Déstruction de l'habitat (Urbanisme, dragage des fonds marins)
\item Changements climatiques (réchauffements, sécheresse)
\item Espèces invasives (liées à la mondialisation ex: frelon asiatiques et abeilles)
\item Surexploitation (surpêche, agriculture intensive)
\item Pollution (Nitrates, phosphates, micro plastiques)
\end{enumerate}

\subsection{Migration des espèces et croissance}

Migrations en cours observées chez de nombreuses espèces, souvent liées aux changements climatiques et la recherche du climat d'origine. La vitesse de ces migrations est souvent trop faible.

\paragraph{Niches climatiques}
\\
\textbf{Définition :} \textit{niche climatique} : 
\\ Les niches climatiques sont des enveloppes climatiques dans lesquelles se développe une espèce particulière. Une niche est ainsi une zone géographique susceptible de se déplacer. On parle aussi de niche fondamentale.

L'étude repose sur une collecte de données (présence, absence) seravnt à l'élaboration d'un modèle statistique pour prévoir l'évolution de la niche, donnat donc lieu à des prédictions spatiales.

\paragraph{Perdants}
Souvent des espèces sensibles aux changements extrêmes (\emph{hêtres, épicéas, disparition des pins à Viège}) qui entrainnent une réduction de leur niche.
\paragraph{Gagnants}
Espèce tirant avantage de ces changement, amenant souvent à une expansion de la niche (\emph{chêne, charme}). En général on constate une progression de la limite sup de la forêt, en Valais par exemple les arbres croissent plus de deux fois plus vite qu'au 19\up{e}.

\subsection{Impact sur la phénologie}

\textbf{Définition :} \textit{phénologie} :
\\ Etude des évenements biologiques cycliques en relation avec les variations saisonnières du climat, très bon indicateur du changement climatique.

\paragraph{}
\textbf{Exemple :}
\\ La floraison du cerisier au Japon, pour laquelle on dispose de beaucoup de données, s'observe normalemnt mi Arvil mais depuis le début du 19\up{e} siècle cette date avance (début Avril fin Mars aujourd'hui).

\paragraph{Indice du printemps} Indice très parlant, on observe une floraison plus précoce, des hivers moins froids et plus courts.
Le printemps avance ainsi d'environ 3 jours par décénie soit une avance environ 15 jours depuis 1975.

Ce décalage provoque un phénomène de \emph{désynchronisation des espèces}, par exemple les oeufs de Mésange ne sont plus pondus pendant les pics d'abondance de larves puisque ces dernières sont sensibles au décalage du printemps.

\paragraph{Maladies et ravageurs}
Les arbres sont également victimes de nombreux parasites comme le bostryche, un coléoptère qui attaque les épiceas affaiblis par la sécheresse. Cette dernière a un double effet puisqu'elle stimule aussi l'insecte, les plus fortes chaleurs accèlerent notamment son cycle de developpement, alors 3 fois plus rapide.
\\ Ainsi on estime que \emph{102 millions} d'arbres on été victimes de la sécheresse en 2016.
Les arbres qui ne dépérissent pas s'acclimatent : on observe de moins en moins de stomates sur les feuilles de ces derniers.


\paragraph{Puits et sources de carbonne}

\textbf{Définition :} 
\paragraph{}
\textit{Puits} : Réservoirs ou composantes du cycle du carbonne qui absorbent plus de carbonne qu'elles n'en émettents : on retrouve les océans (2.3 GT), la biomasse (2.6 GT)
\paragraph{}
\textit{Source} : Composantes du cycle du carbonne qui émettent plus qu'elles n'absorbent, par exemple les énérgies fossiles (7.8 GT)

\paragraph{}
Les plantes jouent un rôles majeur dans le cycle du carbonne, si elles sont responsables des \emph{oscillations intra annulles} du CO$_2$ elles limitent sa présence dans l'atmosphère, qui est de 400ppm actuellement, contre 500ppm estimé sans leur présence.

Les augmentations interannuelles, conséquentes ces dernières années, sont quant à elles dues à l'action de l'Homme.

\paragraph{Boucle de rétroaction}
\textbf{Définition : } \textit{boucle de rétroaction} :
\\ Effet de l'évolution du système sur sa propre évolution, elle peut être :
\begin{itemize}
\item Positives : les effets de l'action travaillent dans le sens de la réaction
\item Négatives : les effets de l'action travaillent contre la réaction
\end{itemize}

\paragraph{}
\textbf{Remarque : } Les boucles rétroactives sont souvent complexes et présentent des aspects positifs et d'autres plus négatifs.
Ainsi par exemple les forêts ont un effet généralement négatif sur le climat, ce qui n'est pas le cas des forêts boréales qui ont un effet rétroactif positif et induisent un réchauffement. 

\subsection{Impact des fôrets sur le climat}
\subparagraph{}
Les fôrets ont un effet sur la température ambiante, elles accumulent la chaleur jour, pour la relacher la nuit.Elles reduisent également la force des vents et mélangent les masses d'air. Elles augmentent l'ombrage ce qui réduit rayonnement solaire et donc le réchauffement au sol.

\subparagraph{}
L'évapotranspiration joue aussi un rôle important dans la régulation du climat. 95\% de l'eau absorbée au niveau des racines est libérée par les stomates, c'est \emph{l'hydrologie}. \emph{L'albedo} a également une grande importance.

\subparagraph{}
Les plantes ont finalment une grande importance dans la régulation de la composition chimique de l'air, elles relachent de nombreuses \emph{COV} qui entrainent la formation de nuages. Par la photosynthèse elles capture aussi une grande quantité de CO$_2$ qui est ensuite stocké dans le sol (principalement les sols arides).

\paragraph{Changement d'albedo en Arctique}
\begin{itemize}
\item La taille de la banquise a diminué de 50\% en 6 ans
\item Le Pergélisol, 24\% des terres exposées dans l'hémisphère Nord subit les effets du réchauffement climatique
\item A terme on risque une libération massive de méthane et de CO$_2$ (de l'ordre de 4 fois les émissions humaines)
\item Les tourbières sont aussi un danger : 3\% surfaces continentales mais 30\% du carbonne des sols mondiaux (puits de carbonne qui devient source)
\end{itemize}

\paragraph{Acidification des océans}
\begin{itemize}
\item En augmentation de 30\% depuis la révolution industrielle
\item Représente une perte du potentielnde pompage : 70\% moins efficace qu'au début de l'ère industrielle
\end{itemize}

\section{Négociations internationnales, justice climatique}
\bigskip
\subsection{Une question de justice}
\bigskip
Au sein du système Terre il y a de nombreux échanges (cycle de l'eau, du carbonne, azote, phosphate, etc.)
\begin{itemize}
\item feedback, rétrocontrôlage
\item instabilité du système tout entier
\end{itemize}
\V

\textit{cf : Global Change and the Earth System (2004)} qui souligne une accélération de \emph{earth system trends} en corrélation avec des phénomènes socio économiques à partir de \textbf{1950}, c'est \textbf{La Grande Accélération}.

\textbf{9} paramètres globaux (dont le changement climatique) définissent les limites planétaires.
Aujourd'hui on est dans une situation diamétralement opposée à toute l'histoire de l'humanité
Les activités humaines ont causé la sortie d'une zone de stabilité relative du climat.
\V

Risques :
\begin{itemize}
\item pour les écosystèmes et les biomes
\item pour les sociétés humaines (augmentation de la zone d'influence de certaines maladies comme la /emph{malaria})
\item économiqiues (fermetures des centrales nucléaires en été)
\end{itemize}
\V

On constate une répartition inégale des risques climatiques (plus élevé pour les pays moins développés). Vulnérabilité, exposition et aléa sont des facteurs de risque.
On a par exemple le nombre de gens exposés à la montée des mers ou encore les rendements agricoles, plus élévé dans le Global North que dans le Global South.
On constate des disparités dans les émissions de CO$_2$, notamment un facteur de \emph{1500} entre le Qatar et le Burrundi pour les émissions CO$_2$/habitant, de plus les USA et l'Europe représentent \emph{50\%} des émissions cumulées dans l'histoire de l'humanité.
\bigskip
\subsection{Notion de justice climatique}
\bigskip
On distingue différents types de justice :
\begin{itemize}
\item \textit{Commutative :} la justice qui règle les échanges, selon le principe de l'égalité arithmétique, entre des personnes elles-mêmes considérées comme égales.
Distribution du gâteau autour de la table.
\item \textit{Procédurale :} Comment distribue-t-on ce gâteau
\item \textit{De reconnaisance :} Chaque personne doit être reconnue autour de la table.
\end{itemize}
\V
En ce qui concerne le climat :
\begin{itemize}
\item Inégalité double vis à vis du ratio émissions/risques
\item Notion de budget carbonne, comment le distribuer ?
\item Coûts de transition énergétique
\item Coûts d'adaptation, de répartition, de compensation
\end{itemize}
\V

Un pays moins développé devrait il avoir droit à une plus grande part du budget carbonne ?
\V

\textit{Schuré : The central point of equity is that it is not equitable to ask some people to surrender necessities so that other people can retain luxuries}

$$\boxed{\textrm{luxes} \neq \textrm{besoins}}$$

Ainsi on peut se demander \textit{comment partager les coûts ?}
\begin{itemize}
\item Principe du \emph{pollueur-payeur} toutefois les plus gros pollueurs sont déjà morts
\item Principe du \emph{bénéficiaire-payeur} qui pose cependant le problème de l'héritage
\item Principe de \emph{capacité à payer} : les plus riches payent ce qui reste injuste d'une certaine façon
\end{itemize}
\bigskip

\subsection{Négociations internationnales sur le climat}
\bigskip
Le réchauffement climatique est un problème d'action \emph{collective}. C'est ce qui rend sa résolution difficile, il nous fautr une certaine coordination.
\V

\textit{La stratégie des biens communs } (G. Hardin) :
\begin{itemize}
\item Une ressource finie et en libre accès
\item L'exploitation de cette ressource est virale
\end{itemize}
\V

Comment réagit un individu rationnel face à cette situation ?

\begin{itemize}
\item MEILLEUR : Je continue, les autres diminuent
\item MOYEN$^+$ : Tout le monde (y compris moi) limiote son exploitation
\item MOYEN$^-$ : Personne ne fait rien, la ressource s'épuise
\item PIRE : Je limite mon exploitation mais personne d'autre n'en fait de même
\end{itemize}
\V

Dans tous les scénarios les principes de rationnalité économique poussent à continuer voir augmenter sa consommation.
\V
\paragraph{Chronologie}
\begin{itemize}
\item 1972 : Stockholm
\item 1987 : Rapport Burtland
\item 1992 : Rio
\end{itemize}
\V

\textbf{GIEC} : Groupe d'experts intergouvernemental sur l'évolution du climat qui vise à synthétiser toute la connaissance humaine sur les grandes questions du climat. \textbf{1988} 3 grands buts : élements physiques, vulnérabilité du système, évaluation des coûts.
\V

\textbf{CNNUCC} : 1992 à Rio ratifié par chaque pays de la planète. Ces textes juridiques contiennent des notions de justice climatique.
\V

\textit{Protocole de kyoto :} 1\up{ère} période d'engagement (1997), il a fallu attendre 2005 pour que le nom entre en vigueur (50\% des pays l'ont ratifié).
\V

Succès en partie dû à la chute de l'URSS, les second accord de Kyoto reste malgré tout.
Accord de Paris : dès 2015, entrée en vigueur en 2016, beaucoup plus souple ils reposent sur trois pilliers :
\begin{itemize}
\item Atténuation
\item Adaptation
\item Financement
\end{itemize}

tragédie des biens communs ressource finie opas essentielle en libre accès et rivale

\subsecion{Climato-scepticisme}

On observe 3 types d'opposition (ordre d'apparition chronologique) : 
\begin{itemize}
\item Rejet de l'existence même du réchauffement climatique
\item Rejet de l'origine humaine du réchauffement climatique
\item Rejet de la gravité du réchauffement climatique

\end{itemize}

\textit{Le climat la grande manipulation, C Gerondeau} (Ingénieur Génie Civil)
\V

\textit{L'imposture climatique, Claude Allègre} (Biochimiste)
\V

Ces publications ont un effet considérable sur l'opinion publique, chute dramatique de la conviction envers le réchauffement climatique entre 2008 et 2010 en Amérique liée à des offensives climatosceptiques motivées par un désir de limiter les effets des traités alors en plein débat (50\% en 2006 35\% deux ans plus tard)
\V

Le mot \emph{scepticisme} relève d'avantage du \emph{négationisme}, il ne s'agit pas là une aptitude saine ou d'une forme de septicisme scientifique puisque les preuves sont nombreuses et accablantes la conclusion des climato sceptiques est souvent d'abord formulée puis les preuves sont niées.
\V

2 aptitudes se démarquent :
\begin{itemize}
\item \textit{Mésinformation :} fonder ses jugements sur un manque d'information ou une inforlmation biaisée ou incomplète
\item \textit{Désinformation :} Biaiser volontairement l'information, le débat, semer le doute (stratégie la plus importante du septicisme climatique) souvent il s'agit d'une vente de service de la part de scientifiques à des lobbys (amiante, pluies acides, etc.) qualifiés de \emph{marchands de doute} \emph{"doubt is our product"}
\end{itemize}
\V

Qui se découpent en 5 caracteristiques \textbf{FACTS} : 
\begin{enumerate}
\item \textbf{F}aux exerts
\item \textbf{A}ttentes irréalistes
\item \textbf{C}hoix selectif des preuves
\item \textbf{T}héorie du complot
\item \textbf{S}ophisme
\end{enumerate}
\V

\begin{enumerate}
\item Quand 97\% scientifiques et des articles s'accordent sur le rôle des gaz à effet de serre certains groupent emploient la stratégie des \emph{faux experts} (souvent spécialisés dans un autre domaine). La désinformation passe aussi par la publication d'articles climatosceptiques \emph{think tank} conservateur dont 90\% sont financés majoritairement par Exxon Koch, ou l'industrie pétrolière en général.
\item Alors que l'on a une confiance très élevée dans le réchauffement climatique, qualifiée \emph{sans équivoque}, \emph{extrêmement probable} 95 à 100\% de confiance,
certains mettent l'accent sur les 5\% d'incertitude et non pas sur les 95 100\%, ne se focalisent pas sur les faits avérés et demandent un 100\% de certitude qui \textbf{N'EST PAS} scientifique
\item On dispose de plus de preuves indépendantes les unes des autres du réchauffement climatique comme la fonte des glaciers, l'augmentation de la hauteur des arbres, un réchauffement de la troposphère, des nuits plus chaudes, etc. 
La stratégie repose sur un choix sélectif des preuves, on ne considère pas l'intégralité des preuves, le résultat est de fait biaisé. 
Par exemple certains avanceront que depuis 1998 le réchauffement de l'air se serait arrété, il faut toutefois regarder l'intégralité de la courbe de plus le réchauffement de l'atmosphère est moindre comparé au réchauffement des océans
\item Climate gate
\item Le sophisme repose sur des raisonnements logiques erronnés et des conclusions hatives, comme des fausses dichotomies, l'utilisation de la réciproque qui n'est pas forcément vraie, etc.
\end{enumerate}
\V

Les raisons sont de 2 types :
\begin{itemize}
\item Biais idéologiques et culturels (USA Démocrates 84\% Républicains 34\% croient au phénomène)
\item Groupes d'intérêts (économiques)
\end{itemize}
\bigskip

\section{Solutions pragmatiques}
\bigskip
Le défi : freiner le changement climatique, requiert de réduire les émissions de CO$_2$, notamment bruler moins d'énergie fossile
\V

\emph{Adaptation} pour diminuer les conséquences des effets du changement climatique, on a toutefois besoin d'\emph{attenuer} les effets de ce dernier car une adaptation totale est impossible
\V

On considère dans le modèle actuel que les dommages ne dépendent pas des émissions mais plutôt de la concentration de particules, en prenant également en compte l'accumulation (c'est à dire la somme des émissions passées)
\V

D'où viens la \textbf{limite des 2\textdegree C} ?

On cherche ainsi à minimiser les dommages dus aux émissions ainsi que le coût d'atténuation. On cherche à minimiser une fonction prenant en paramètres les coûts économiques ainsi que les coûts écologiques et humains (plus difficiles à quantifier, liés aux effets des dommages sur l'environnement) il ne s'agit \textbf{pas de 100\%} d'atténuation on parle \emph{d'optimum économique}.
Le niveau optimal d'adaptation peut être défini, il n'annule pas nécessairement tous les dommages.

\subsection{Adaptation}
\bigskip

\textit{L'atténuation} coûte pour celui qui réalise les efforts et profite à tous même à ceux qui ne font rien, cela pose un problème dans un contexte de compétition commerciale et de globalisation.
Un pays possède de plus un \emph{avantage privé à émettre}, les coûts sont pourtant collectifs, cela encourage encore d'avantage un comportement égoïste vis à vis du réchauffement climatique.
Elle incite de plus un \emph{déplacement de la production} vers les pays où les normes sont les plus permissives, ou l'atténuation est plus fainle donc, ce qui a un double effet augmentant les émissions dues à un transport qui aurait pu être évité (on parle de \emph{carbon leakage}) et freine la mise en place de mesures climatiques globales (avec l'émergence de véritables \emph{pollution heaven}.
\V

\textit{L'adaptation} peut nuire par contre aux pays voisins (un pays qui canalise les cours d'eau en amont assèche les pays en aval) tout en profitant au pays qui la met en place. Sans coopération entre les pays et dans un contexte compétitif c'est naturellement cette alternative qui serait mise en avanserait mise en avant. 
Afin de minimiser les dommages il faut pourtant un \textbf{mélange} des deux.
\V

Consiste en partie en des mesures de protection:
\begin{itemize}
\item Occupation du territoire : 
\subitem quitter les régions côtières, digues, barrages
\item Modification de nos pratiques : 
\subitem agriculture: changer de cultures, réservoirs et irrigation
\subitem tourisme: réorienter l'offre
\item Assurance :
\subitem diversification de nos activités, partage des risques
\subitem assurances, weatherderivatives, catbonds

L'accumulation des gaz à effet de serre est d'autant plus grave que la urée de vie des ces derniers est très importante :

le CO$_2$ 100 ans, le CF$_4$ 50 000 ans, ce phénomène accentue donc l'effet d'accumulation. Cumulé avec l'nertie du système, notamment les boucles rétroactives et l'irréversibilité des effets, on se retrouve face à un dilemme, on ne peut dépenser des sommes astronomiques pour l'atténuation, mais ce qu'on ne dépense pas aujourd'hui coûtera deux fois plus cher demain.
\V

Un autre facteur rend la résolution de ce problème difficile, ses paramètres sont en constante évolution (croissance démographique, évolutions technologiques, par exemple). On considère donc :
\begin{itemize}
\item \emph{Avantages} à retarder les mesure :
\subitem Laisser le temps à la recherche de trouver des solutions permettant de réduire les émissions à faible coût (progrès technique)
\subitem Laisser augmenter les revenus pour pouvoir mieux supporter les efforts
\item \emph{Désavantages} à retarder les mesures :
\subitem Si l'on continue de construire des installations durables émettrices de CO2 (centrales électriques à charbon, nos voitures), il sera plus coûteux d'y renoncer
\subitem Iniquité, imposer aux suivants de faire les efforts
\end{itemize}
\V

D'un point de vue \textbf{éthique}, il est aussi difficile d'attribuer un montant aux dégradation de l'environnement mais aussi et surtout aux vies humaines et à la qualité de viue quotidienne. Plusieurs solutions se présentent pour la baisse des émissions : une diminution de la population, une baisse de la production et de la consommation, une transition énergétique.
\V

\textit{Progrès technique :} Recherche invention, innovation, il s'agit d'une invention et de sa mise en avant. Le progrès technique n'est pas forcément favorable au climat (par exemple la capacité à puiser toujours plus loin les énergies fossibles, forages pour les gaz de schistes, forages pétroliers en Arctique, mais aussi une hausse de l'utilisation de moteurs donc de la consommation). Certains aspects du progrès lui sont quand même favorables avec notamment le développement de machines à plus faible consommation. Il est aussi favorable à la stabilisation du climat (énergie hydraulique ou nucléaire mises en avant).
\V

Pour que le budget climat soit respecté il faut que tous les moteurs soient remplacés, ce processus devrait d'abord passer par une interruption de la production de tels engins.

La limitation de la mise en place d'alternatives plus vertes n'est pas un manque de savoir (en terme d'innovation) mais de vouloir (rendements plus faibles, coûts plus élevés). L'Etat pourrait jouer un rôle en taxant les alternatives les plus mauvaises pour l'environnement.
\V

\textit{Identité de Kaya :}
$$\boxed{\textrm{émissions de CO$_2$} = \textrm{Pop} * \frac{\textrm{PIB}}{\textrm{Pop}} * \frac{\textrm{E}}{\textrm{PIB}} * \frac{\textrm{CO}_2}{\textrm{E}}}$$
\V

Entre 2010 et 2015 la population a uagmenté de 1.2\%, le  PIB par habitant de 1.6\%, provoquant une hausse de 1.2\% des émissions malgré une baisse de 1.4\% de l'utilisation d'énergie par le PIB.

Autre exemple, le \textit{transport aérien :} le progrès technique induit une baisse de 60\% de la consommation en kérosène mais une augmentation de 250\% du nombre de passagers par km.
On parle d'\textbf{effet rebond} : la baisse de la consommation induit une baisse du prix qui induit elle même une hausse de la consommation.

\subsection{Une approche nationale pragmatique}
\bigskip

La politique climatique apparait comme une stratégie win/win, elle présente de nombreuses retombées positives comme une incitation à l'innovation, la création de nouveaux marchés et d'emplois. On parle de \emph{croissance verte}.
\V

En 1990 les émmissions de gaz à effet de serre en Suisse sont en majorité dues à la production énergétique, les emmissions domestiques et les processus industriels.
Le point de depart de la politique climatique Suisse en \textbf{1973} :
Le pays était dépendant de l'importation d'énergies fossiles depuis des pays instables (pas motivé par climat en premier lieu).
Cela se fait par la mise en place de programmes énergétiques : \emph{Energie 2000 SuisseEnergie} l'accent est mis sur l'augmentation de l'efficacité énergétique, rien de contraignant, moyens softs comme des recommendations, des conseils, la mise en place d'un budget, des étiquettes énergie (même s'il faut relativiser ce dernier point par exemple un \emph{Randge Rover} très polluant étiquetté A...).
\V

Au niveau des \textit{bâtiments :}

La taxe CO$_2$ fut introduite dans un second temps, elle est prélevée sur les combusitibles (et non les carburants) utilisés à des fins énergétiques au prorata de leur contenu en carbone pour
 motiver par exemple les particuliers à opter pour un remplacement de chauffage plus écologique (atténuation).
Le taux de la taxe triple entre 2008 et 2012, la recette est redistribuée aux contribuables au 2 tiers, le tiers restant est dédié au programme bâtiment qui offre des subventions pour la construction de bâtiments moins gourmands en énergitiments moins gourmands en énergie. 
Toutefois l'effet dissuasif de la taxe CO$_2$ est souvent masqué par la variation du prix du brut
Ces mesures sont toutefois vagues et nombreux sont les moyens de les contourner.
\V

Il y a ensuite des mesures dans le domaine des \emph{transports :} 

\emph{Ecodrive} pour apprendre à conduire en consommant moins promet une diminution de 10 à 15\%, la motivation est aussi économique pour le conducteur.
On peut également citer la promotion du \emph{CarSharing} et des transports publiques.
Il y a aussi une \emph{redevance poids lourds} liée aux prestations.
Le \emph{centime climatique} (1.5ct par litre d'essence) versé à une fondation privée subventionne la réduction des émmssions est une compensation obligatoire pour les importateurs de carburant.
A cela s'ajoute une prescription concernant les voitures de tourisme neuves (imposant un maximum d'émmission de 130g de CO$2$ par km, ce chiffer sera poussé à 95g en 2023)
\V

La stabilisation des émissions constatée ces dernières années est à nuancer, elle est en partie obtenue en important de plus en plus de produits riches en carbone, émissions dues à la consommation et moins la production.

\end{document}
