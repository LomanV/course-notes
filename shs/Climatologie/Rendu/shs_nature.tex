\documentclass[12pt, letterpaper, twoside]{article}
\usepackage[top=3cm, bottom=3cm, left=2cm, right=2cm]{geometry}
\usepackage[francais]{babel}

\begin{document}
\section{Enjeux naturels}
\subsection{Developpement sur l'ingénierie chimique et ses impacts environnementaux}
\bigskip
L'ingénierie chimique des aérosols qui sont déstinés à être envoyés dans l'asténosphère est un enjeu capital, tant du point de vue technique que du point de vue environnemental. Ces derniers sont conçus pour résider dans l'asténosphère, plus éloignée que la troposphère, augmenatant leur durée de vie.
\subsubsection{Le point de vue technique}
Du point de vue technique tout d'abord, les propriétés physico-chimiques des particules vont influer sur leur \emph{efficacité} mais également sur leur \emph{durée de vie} dans l'asténosphère. 
\ \linebreak
Les propriétés \textbf{physiques}: 
\begin{itemize}
\item les particules plus petites et légères, à partir de 10 $\mu$m, retombent plus rapidement.
\item les particules les plus petites sont plus efficaces que les plus lourdes et ont une absorption plus faible des rayons infrarouges.
\end{itemize}
La taille des particules en suspension ainsi que leur masse ont donc un effet conséquent sur le climat.
\ \linebreak
A cela s'ajoutent les propriétés \textbf{chimiques}:
\begin{itemize}
\item toutes les particules ne réfléchissent pas de la même façon les rayons.
\item la particule doit être peu polluante.
\item toutes les particules n'ont pas la même durée de vie en suspension, le SO$_2$ par exemple a une durée de vie estimée à \emph{30 jours} contre le puit le plus efficace, alors que le SO$_4$ a une durée de vie de \emph{2 ans}.
\end{itemize}
Il faut de plus prendre en compte un grand nombre de paramètres complexes et parfois difficilement prévisibles comme l'humidité, les vents et la température. Une étude de ses paramètres doit de plus s'appuyer non pas seulement sur des modèles globaux mais également sur des modèles régionnaux puisque les particules en suspens ne se comportent pas de la même façon dans différentes régions du monde, on anticipe par exemple que ces dernières auraient une durée de vie moindre au niveau des pôles. 
\ \linebreak
Toutes ces considérations rendent la prévivision des effets d'une telle solution difficile à établir, selon les modèles on estime nécessaire une injection de 5\footnote{observation de Crutzen (2006) par analogie avec les effets de l'éruption du Pinabuto} à 1.5\footnote{estimation de Rasch basée sur un modèle climatique couplé plus récent et réaliste} TgS de SO$_2$ par an pour compenser le réchauffement dû à un doublement du taux de CO$_2$ dans l'atmosphère. Les études s'accordent sur le fait qu'une source de l'ordre de 15 à 30 fois les sources actuelles non volcaniques serait nécessaire pour compenser les effetsvdu réchauffement climatique. Enfin certains argumentent que l'importante perturbation du budget sulfate induite dans la stratosphère ne serait que faible à l'échelle de toute l'atmosphère. 
\subsubsection{L'aspect environnemental}
Le choix des particules à injecter lors du processus est également pensé pour minimiser l'impact sur l'environnement. Si les sulfates sont toujours présents dans l'atmosphère, l'utilisation de ces derniers pourrait perturber le cycle du \emph{sulfure}, contribuer à l'apparition de \emph{pluies acides} à l'appauvrissement de la couche d'ozone. En plus d'augmenter l'albedo une hausse de la concentration des sulfates dans l'stratosphère pourrait influer sur la taille des nuages et leur persistance en augmentant la condensation. Toutefois par analogie avec les éruptions volcaniques ces effets secondaires, tout comme l'effet principal du refroidissement global, ne sont que temporaires dans la mesure d'injections occasionnelles.
\end{document}
