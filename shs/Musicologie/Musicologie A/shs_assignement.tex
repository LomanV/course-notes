\documentclass[12pt, letterpaper, twoside]{article}
\usepackage[top=2cm, bottom=2cm, left=2.5cm, right=2.5cm]{geometry}
\usepackage[francais]{babel}
\usepackage[utf8]{inputenc}

\date{\vspace{-5ex}}
\usepackage{fancyhdr}
\usepackage{setspace}

\begin{document}
\thispagestyle{fancy}
\noindent
\textbf{Musicologie Histoire de l’improvisation classique} \hfill \textbf{Vezin Lom\`an} \\
\normalsize Prof.\ Constance Frei \hfill Date de rendu: 12/11/19
\begin{center}
\textbf{La virtuosité}\\
\end{center}
\begin{spacing}{1.5}

`\emph{Un virtuose nous fait entendre la musique non pas comme elle est écrite, mais comme il la sent}' affirmait le cinéaste français Robert Bresson. Le sens du mot `virtuosité' a su évoluer au fil des traditions et innovations techniques en faisant un témoin important de l'histoire des arts et plus particulièrement de la musique et de l'improvisation musicale. 
Soulignons que la virtuosité n'est pas une création du XIXe siècle, comme l'attestent les recherches de Lowinsky on retrouve en Europe dès la Renaissance l'ébauche d'une recherche grandissante de la virtuosité chez l'interprète comme chez l'auditeur.
La virtuosité est premièrement le témoin d'un génie et d'une reconnaissance qu'on accorde aux Hommes des beaux arts et des hautes sciences se distinguant dans leur domaine par leur professionnalisme. D'origine italienne ce mot désigne celui qui a de la `\emph{virtu}', qui fait preuve d'une certaine qualité morale. Ce n'est que plus tard que le mot est rattaché à l'esthétique artistique, et notamment musicale lorsqu'il arrive en France à la fin du XVIIe. 
Dans cet esprit, Kuhnau associe au XVIIIe la virtuosité à l'excellence technique et théorique, pour lui interprète et compositeur sont indissociables, plus qu'un simple executant, le virtuose compose ses \oe{}uvres.
En rupture avec les m\oe{}urs moyenâgeuses, l'Homme de la Renaissance manifeste son individualisme en valorisant son expression personnelle. Une hiérarchie se crée, de nombreux évènements, presque sportifs, en témoignent, comme les jeux gratuits ou les improvisations lors de joutes entre interprètes, variant sur diverses airs et thèmes. On peut penser aux duels opposant Liszt à Thalberg lesquels improvisaient sur des fantaisies. Ces évènements prônent la démonstration technique, ponctués de défis toujours plus audacieux, citons Mozart qui jouait sans même voir le clavier. Dans ce contexte l'improvisation comme jeu social est alors un témoin de la virtuosité, ainsi qu'un moyen de surprendre le public et susciter son admiration. On ne peut s'empêcher aujourd'hui de penser à John Cage qui à l'instar de ses prédécesseurs cherche par sa technique et la surprise sonore de son piano arrangé à décontenancer son auditoire. 
Le `\emph{virtuoso}' cherche aussi à transmettre par sa voix ou son instrument quel qu'il soit sa sensibilité émotionnelle à son public. D'abord honorifique, le mot prend un sens péjoratif à l'orée du XIXe. On reproche notamment aux chanteurs d'opéra, plus particulièrement d'\emph{opera seria}, de ne pas assez respecter le travail du compositeur au profit d'une excellence technique poussée à l'extrême. 
Dans un autre contexte, certains virtuoses instrumentaux cependant font l'objet de louanges. Interprètes et compositeurs au sens de Kuhnau, ces derniers improvisent sur leurs propres sonates et concertos en rajoutant des ornements et des cadences rapides. Chaque production est alors un embellissement et une redécouverte de l'\oe{}uvre, une recréation du texte et une participation dynamique de l'interprète comme du public. On improvise aussi sur les préludes et les variations, l'improvisation est alors le support de l'expression d'une spontanéité et d'une passion propres à la virtuosité. L'interprétation des compositions devient également un aspect important de la virtuosité. Dans ce registre de nombreux arrangements modernes d'\oe{}uvres plus anciennes voient le jour, citons Geminiani et ses arrangements des \oe{}uvres de  Vivaldi. 
La virtuosité au XIXe, en tant que recherche constante de la progression dans la maitrise technique, se nourrit des innovations technologiques des instruments. Apparu au milieu du XVIIIe, le piano se développe et sa popularité grandissante multiplie les virtuoses. Le facteur du piano, organisateur de concerts et impressario, voit son art en plein essor, la commercialisation s'étend à une vaste échelle. En réponse, le piano adopte une forme carré plus pratique et des noms comme Broadwood ou Stodart se démarquent dans le domaine. 
Si la mécanique viennoise s'impose tout d'abord dans le domaine associée à Mozart par exemple, c'est à Londres que se développe principalement la facture du piano. On vante l'ampleur de la sonorité et la solidité des pianos anglais. Parmi les avancées technologiques notables citons l'échappement Stein qui permet une vibration libre de la corde lorsque la touche reste enfoncée apportant une certaine lourdeur à la couleur musicale, suivi du double échappement d'Erad qui répond à un objectif de vélocité soutenue et maintenue durant toute une pièce dont Liszt fait grandement usage. Témoin d'un progrès croissant depuis Mozart, le piano moderne est né, point de départ d'une gymnastique nouvelle chez l'interprète, citons Paganini qui repousse les limites connues des performances techniques de l'instrument. 
La virtuosité des pianistes devient progressivement la publicité des facteurs. Cette relation très commercialisée a des conséquences remarquables, enfermant le virtuose dans un conformise certain. Le pianiste virtuose devient de plus en plus exclusivement compositeur, notamment d'\oe{}uvres jugées plus simples destinées à la vente, la position de virtuose est au centre de plus en plus de critiques et le rapport à l'improvisation se tarit. 

\end{spacing}

\end{document}
