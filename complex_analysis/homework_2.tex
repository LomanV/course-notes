\documentclass[12pt]{article}
\usepackage[top=1in, bottom=1in, left=1in, right=1in]{geometry}
\usepackage[french]{babel}

\usepackage[onehalfspacing]{setspace}

\usepackage{amsmath, amssymb, amsthm}
\usepackage{enumerate, enumitem}
\usepackage{fancyhdr, graphicx, proof, comment, multicol}
\usepackage[none]{hyphenat}
\usepackage{dirtytalk}
\binoppenalty=\maxdimen
\relpenalty=\maxdimen

\usepackage{microtype}
\usepackage{mathpazo}
\usepackage{mdframed}
\usepackage{parskip}
\linespread{1.1}
\usepackage{graphicx}
\usepackage{subfig}

\usepackage{mathrsfs}
\usepackage{amsfonts}
\usepackage{amsmath}
\usepackage{amssymb}

\usepackage{mathtools}
\newcommand{\defeq}{\vcentcolon=}
\newcommand{\eqdef}{=\vcentcolon}

\newenvironment{statement}[1]
{\begin{mdframed}[linewidth=0.6pt]
        \textsc{Théorème #1:}

}
    {\end{mdframed}}

\newcommand{\R}{\mathbf{R}}
\newcommand{\C}{\mathbf{C}}
\newcommand{\Z}{\mathbf{Z}}
\newcommand{\N}{\mathbf{N}}
\newcommand{\Q}{\mathbf{Q}}

\newcommand{\de}{\mathrm{d}}

\begin{document}
        \noindent
\textbf{Analyse Complexe} \hfill \textbf{Vezin Lomàn}\\
\normalsize MAT431 \hfill Date de rendu: 13/11/2020\\

\begin{center}
\textbf{Devoir maison 2}
\end{center}

\begin{statement}{1}
        Les conditions suivantes sont équivalentes pour une application $f \in \mathcal{C}^{2}(\R^{n}, \R^{n})$ 
        \begin{align*}
                \star \; & f \; \text{est propre et } df_{x} \; \text{est inversible pour tout } x \in \R^{n} \\
                \star \star \; & f \; \text{est un difféomorphisme } \mathcal{C}^{2} \; \text{de } \R^{n} \; \text{dans lui même} 
        .\end{align*}
\end{statement}

\begin{proof}
        \textbf{1.} On montre que $\star\star$ implique $\star$. Comme $f$ est un $\mathcal{C}^{2}$ difféomorphisme $f^{-1}$ est en particulier continue, ainsi pour tout compact $K \subset \R^{n}$, $f^{-1}(K)$ est compact et donc $f$ est propre. Fixons $x \in \R^{n}$, comme \[
                f^{-1}\circ f = Id
        ,\] remarquons que la formule de dérivation d'une fonction composée nous donne \[
        d(f^{-1})_{f(x)} df_{x} = I_{n}
,\] et ainsi $df_{x}$ est inversible. 

\medskip

Supposons à présent et pour le reste de l'exercice $\star$. Montrons dans un premier temps que $f$ est surjective. Pour ce faire montrons que $f(\R^{n})$ est un ouvert fermé de $\R^{n}$, comme cet ensemble est non vide et que $\R^{n}$ est connexe on en déduira $f(\R^{n}) = \R^{n}$.

Soit $y_0 \in f(\R^{n})$ et $x_0 \in \R^{n}$ satisfaisant $f(x_0) = y_0$. Par hypothèse nous savons que $df_{x_0}$ est inversible, par le théorème d'inversion locale il existe donc deux voisinages ouverts $U \subset \R^{n}$ et $V \subset f(\R^{n})$ contenant respectivement $x_0$ et $y_0$ tels que \[
        f(U) = V \subset f(\R^{n})
.\] Nous venons de montrer que $f(\R^{n})$ est ouvert. Montrons à présent qu'il est fermé.


Soit $(y_{k})_{k\in\N}$ une suite dans $f(\R^{n})$, convergente avec pour limite $y$, et soit $(x_{k})_{k\in\N}$ une suite de $\R^{n}$ satisfaisant \[
        f(x_{k}) = y_{k} \quad \forall k \in \N
.\] 
L'ensemble  \[
K \defeq \{y_{k}, \; k\in \N\} \cup \{y\} \subset \R^{n}
\] est compact puisque fermé et borné par convergence de $(y_{k})_{k}$ vers $y$. Comme nous avons supposé $f$ propre $f^{-1}(K)$  est aussi compact et contient chacun des $x_{k}$ pour $k \in \N$. Nous pouvons alors exhiber une sous suite convergente vers une limite que nous pouvons noter $x$  \[
x_{k_{j}} \underset{j\to\infty}{\longrightarrow} x
.\] 


Par continuité de $f$ et comme la convergence de la suite $y_{k}$ entraine la convergence vers la même limite de la sous suite extraite $y_{k_{j}}$ on obtient
\begin{align*}
        y = \lim_{j\to \infty}y_{k_{j}} &= \lim_{j\to \infty}f(x_{k_{j}}) \\
                                        &= f(\lim_{j\to \infty}x_{k_{j}}) \\
                                        &= f(x)
.\end{align*}

Ainsi $y \in f(\R^{n})$ qui est donc fermé. Nous pouvons conclure.

\bigskip

\textbf{2.} On étudie à présent l'injectivité de $f$. Fixons $z \in \R^{n}$ et considérons l'ensemble
\begin{align*}
        S_{z} &\defeq \{x \in \R^{n} \;|\; f(x) = f(z)\} \\
              & \; = \{x \in \R^{n} \;|\; g(x) = 0\} 
,\end{align*} où $g$ est la fonction auxiliaire donnée par
 \begin{align*}
         g : \R^{n} &\longmapsto \R^{n} \\
         x &\longmapsto f(x) - f(z)
.\end{align*}

Puisque $f$ est propre on en déduit directement que $g$ l'est aussi. Ainsi $S_{z}$ est compact comme préimage du singleton $\{0\}$ lui même compact.
De plus on voit facilement que $f$ et $g$ ont la même différentielle. Si par l'absurde $S_{z}$ disposait d'un nombre infini de points il contiendrait un point d'accumulation, $a$ disons. Comme $dg_{a}$ est inversible nous pouvons appliquer à nouveau le théorème d'inversion locale pour trouver un voisinage ouvert $U$ de $a$ tel que \[
        g : U \longmapsto g(U)
\] soit une bijection. Soit $(x_{k})_{k\in \N}$ une suite de $S_{z}$ convergeant vers $a$, on peut trouver au moins un $x_{i}$ de cette suite dans $U$ mais alors  \[
g(x_{i}) = 0 = g(a)
\] comme $x_{i}, a \in S_{z}$, ce qui contredit l'injectivité de la restriction de $g$ à $U$. $S_{z}$ est donc nécessairement fini.

\bigskip

\textbf{3.} Posons \[
        X(x) \defeq (dg_{x})^{-1}g(x)
,\] et considérons l'équation différentielle donnée par \[
\mathcal{E}_{z} : \quad \begin{cases}
        x'(t) = -X(x(t)) \\
        x(0) = x_0 \in \R^{n}
\end{cases}
.\] 

\textit{a.} Comme $f$ est $\mathcal{C}^{2}$, $g$ l'est aussi et $X$ est $\mathcal{C}^{1}$. Par Cauchy $\mathcal{E}_{z}$ admet pour un $\alpha > 0$ une solution maximale $x$ sur $[0, \alpha[$. Montrons que $\alpha = +\infty$ en utilisant le lemme des bouts (Théorème VIII.3.9). Soit $t \in [0,\alpha[$, on a
\begin{alignat*}{3}
        \frac{\de}{\de t}(g(x(t))) &= dg_{x(t)}x'(t), && \text{par la formule de dérivation composée}\\
                                   &= -dg_{x(t)}(dg_{x(t)})^{-1}g(x(t)), \quad && \text{comme } x \text{ est solution} \\
                                   &= -g(x(t))
.\end{alignat*}
On en déduit que \[
        g(x(t)) = e^{-t}g(x_0)
,\] et donc comme $t > 0$ on obtient  \[
\|g(x(t))\| \le \|g(x_0)\|
.\] Comme $g$ est propre $g^{-1}(\overline{B(0, \|g(x_0)\|)})$ est compact et comme \[
x(t) \in g^{-1}(\overline{B(0, \|g(x_0)\|)}), \quad \forall t \in [0,\alpha[
,\] par le lemme des bouts $\alpha = +\infty$.

\medskip

\textit{b.} Notons $S_{z} = \{z_1, \ldots, z_{m}\}$. Remarquons dans un premier temps par définition de $S_{z}$ que \[
        g(z_{i}) = 0, \quad \forall i \in \{1, \ldots, m\} 
.\] Par hypothèse $df_{z_{i}}$ est inversible et nous avons montré que $dg_{x} = df_{x} \; \forall x \in \R^{n}$. On en déduit par le théorème d'inversion locale l'existence d'un $\delta_{i} > 0$ et d'un voisinage $V$ de 0, tel que la restriction de $g$
\begin{align*}
        g : B(z_{i}, \delta_{i}) \longmapsto V
\end{align*} soit un difféomorphisme.

Soit $y \in V$ et $x_0 \defeq g^{-1}(y)$. Considérons l'application
\begin{align*}
        h : t \longmapsto g^{-1}(e^{-t}y)
,\end{align*} montrons qu'il s'agit d'une solution, on pourra conclure par unicité.
En effet $h(0) = g^{-1}(y) = x_0$ et
\begin{align*}
        h'(t) &= d(g^{-1})_{e^{-t}y}(-e^{-t}y) \\
              &= -d(g^{^{-1}})_{g(h(t))}g(h(t)) \\
              &= -(dg_{h(t)})^{-1}g(h(t)) \\
              &= -X(h(t))
.\end{align*}

On obtient finalement par unicité de la solution que \[
        \lim_{t\to \infty}x(t) = \lim_{t\to \infty}h(t) = \lim_{t\to \infty}g^{-1}(e^{-t}y) = g^{-1}(0) = z_{i}
.\] Puisque le choix de $y \in V$ et donc de $x_0 \in B(z_{i}, \delta_{i})$ était arbitraire on a le résultat voulu. 

\medskip

\textit{c.} Soit $x_0 \in \R^{n}$ et $x$ une solution de $\mathcal{E}_{z}$ satisfaisant $x(0) = x_0$. Par \textit{a.} $x$ reste dans un compact pour tout $t > 0$ donc l'image de $[0, +\infty[$ par $x$ doit avoir au moins un point d'accumulation, disons $l$. Soit donc $(t_{j})_{j\in\N}$ une suite strictement croissante, $t_{j} \to_{j} \infty$, telle que \[
        x(t_{j}) \underset{j\to \infty}{\longrightarrow} l
.\] 
Par le point \textit{b.} $g(x(t)) = e^{-t}g(x_0)$, on en déduit donc par passage à la limite \[
        g(l) = \lim_{j\to \infty}g(x(t_{j})) = \lim_{j\to \infty}e^{-t_{j}}g(x_0) = 0
.\] Ainsi $l \in S_{z}$ et donc $l = z_{i}$ pour un $i \in \{0, \ldots, m\}$.

Par convergence de la suite $(x(t_{j}))_{j}$, pour le $\delta_{i}$ du point \textit{b.} on peut trouver $J > 0$ suffisamment grand de sorte que \[
x(t_{j}) \in B(z_{i}, \delta_{i}), \; \forall j \ge J
.\]

En particulier par le point \textit{b.} la solution $h$ satisfaisant  $h(0) = x(t_{J})$ converge vers $z_{i}$. Remarquons que la solution \[
        \tilde{h}(t) = x(t + t_{J})
,\] satisfait également $\tilde{h}(0) = x(t_{J})$ et donc par unicité globale on obtient \[
x(t + t_{J}) = h(t) \quad \text{et} \quad \lim_{t\to \infty}x(t) = z_{i}
.\]  

\medskip

\textit{d.} Chaque $A_{i}$ est bien défini par unicité globale des solutions, garantie par la proposition VIII.3.1 du cours\footnote{C'est de cette proposition dont on se sert tout du long pour l'unicité globale}, puisque $\mathcal{C}^{1}$ implique localement Lipschitz.

Soit $i \in \{1, \ldots, m\}$ et soit $x_0 \in A_{i}$. Soit $x$ la solution de condition initiale  $x_0$. En reprenant le $\delta_{i}$ du point \textit{b.} par définition de $A_{i}$ il existe un $T > 0$ tel que pour tout $t \ge T$ \[
        |x(t) - z_{i}| \le \frac{\delta_{i}}{2}
.\] Comme les solutions sont continues par rapport au conditions initiales, il existe $\epsilon > 0$ tel que si $|x_0 - y_0| < \varepsilon$ et si $y$ est solution de condition initiale $y_0$ \[
|x(t) - y(t)| \le \frac{\delta_{i}}{2} 
.\]  
On conclut par inégalité triangulaire \[
        |y(t) - z_{i}| \le |y(t) - x(t)| + |x(t) - z_{i}| \le \delta_{i}
.\]
Par le point \textit{b.} il en découle que \[
        \lim_{t\to \infty}y(t) = z_{i}
.\] Ainsi $x \in A_{i}$. Comme le choix de $i$ et de $x_0$ étaient arbitraires, nous venons de montrer que $A_{i}$ est ouvert pour tout $i \in \{1, \ldots, m\}$ 

\medskip

\textit{e.} On a par le point \textit{c.} que \[
\R^{n} = \bigcup_{i=1}^{m}A_{i}
.\] Par le point \textit{d.} nous savons de plus que chaque $A_{i}$ est un ouvert non vide de $\R^{n}$. Il est clair par unicité des limites dans $\R^{n}$ que les $A_{i}$ sont deux à deux disjoints. Par connexité de $\R^{n}$ on doit donc nécessairement avoir $m = 1$. Ainsi $g$ ne s'annule qu'en un unique point, c'est à dire par définition de $g$ qu'il existe un unique $x \in \R^{n}$ tel que $f(x) = f(z)$. Puisque le choix de $z \in \R^{n}$ était arbitraire nous avons bien montré l'injectivité de $f$.

Nous pouvons en conclure avec le point \textbf{1.} que $f$ est une bijection de $\R^{n}$ dans lui même et cela conclut la preuve.
\end{proof}

\bigskip

\textbf{4. Une application} \textit{a.} Soit $g : \R^{n} \longmapsto \R^{n}$ continue avec  
\begin{equation}
        \lim_{\|x\|\to \infty}\|g(x)\| = +\infty.
\end{equation}
Soit $K \subset \R^{n}$ un compact. Par Borel Lebesgue $K$ est fermé et borné. Par continuité de $g$, $g^{-1}(K)$ est fermé. Par un argument ensembliste on a de plus \[
g(g^{-1}(K)) \subset K
.\] En particulier comme $K$ est borné, par (1) $g^{-1}(K)$ doit aussi être borné. $g^{-1}(K)$ est donc fermé borné, par Borel Lebesgue c'est donc un compact et $g$ est propre. 

\medskip

\textit{b.} Par équivalence des normes sur $\R^{n}$ nous pouvons considérer la norme 2 afin de faciliter les calculs. L'application
\begin{align*}
        \|.\|^{2} : \R^{n} &\longmapsto \R^{n} \\
        x &\longmapsto \langle x, x \rangle = \sum_{i=1}^{n}x_{i}^{2}
\end{align*} est $\mathcal{C}^{\infty}$. L'application induite par $A \in GL_{n}(\R)$ est clairement $\mathcal{C}^{\infty}$ On en déduit que $f$ est $\mathcal{C}^{\infty}$ comme somme de compositions de fonctions $\mathcal{C}^{\infty}$.


$f$ est en particulier continue, nous appliquons le critère précédent pour montrer qu'elle est propre. Dès que $\|x\| > \sqrt{2}$ on a \[
        \varphi(\|x\|^{2}) = 0 \quad \text{donc,} \quad f(x) = x
.\] Ainsi \[
\lim_{\|x\|\to \infty}\|f(x)\| = +\infty
,\] et $f$ est propre. 

Soit $x \in \R^{n}$, on calcule à présent la différentielle $df_{x}$ de $f$ en $x$.
\end{document}
