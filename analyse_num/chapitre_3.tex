\documentclass[12pt]{book}
\usepackage[top=1in, bottom=1in, left=1in, right=1in]{geometry}
\usepackage[onehalfspacing]{setspace}
\usepackage[french]{babel}
\usepackage[T1]{fontenc}
              
\usepackage{amsmath, amssymb, amsthm}
\usepackage{mathrsfs}
\usepackage{amsfonts}
\usepackage{mathtools}
\usepackage{stmaryrd}

\newcommand{\defeq}{\vcentcolon=}
\newcommand{\eqdef}{=\vcentcolon}
                   
\usepackage{enumerate, enumitem}
\usepackage{fancyhdr, graphicx, proof, comment, multicol}
\usepackage[none]{hyphenat}
\usepackage{dirtytalk}
\usepackage{proof}
           
\usepackage{graphicx}
\usepackage{tikz-cd}
\usepackage[all]{xy}

\usepackage{centernot}
\usepackage{mathtools}
\usepackage{ stmaryrd }

\usetikzlibrary{decorations.pathmorphing}
\usetikzlibrary{decorations.markings}
\usetikzlibrary{arrows.meta,bending}

\usepackage{import}
\usepackage{xifthen}
\usepackage{pdfpages}
\usepackage{transparent}
\newcommand{\incfig}[1]{%
    \def\svgwidth{\columnwidth}
    \import{./figures/}{#1.pdf_tex} 
}
            
\newtheorem{lemma}{Lemme}[section]
\newtheorem{theorem}[lemma]{Théorème}
\newtheorem{cor}[lemma]{Corollaire}
\newtheorem{prop}[lemma]{Proposition}
                    
\theoremstyle{definition}
\newtheorem{definition}[lemma]{Définition}
\newtheorem{example}[lemma]{Exemple}
\newenvironment{comments}{}{}
\newtheorem*{notation}{Notation}

\theoremstyle{remark}
\newtheorem*{remark}{Remarque}


\newcommand*\quot[2]{{^{\textstyle #1}\big/_{\textstyle #2}}}

\begin{document}
	\chapter{Interpolation de fonctions}
	Étant donné une fonction $f \in \mathcal{C}^{0}([a,b])$, on cherche un polynôme $p_m$ de degré $m$ qui approche au mieux $f$.
	\begin{theorem}[Théorème d'approximation de Weierstrass]
		Pour toute fonction continue sur un segment $f$ et tout entier $n$ il existe un polynôme $p_{n}$ de degré inférieur ou égal à $n$ tel que  \[
		\lim_{n \rightarrow +\infty} \|f-p_{n}\| = 0
		.\] 
	\end{theorem}
	\begin{remark}
		Pour toute fonction continue il existe une suite de polynôme qui converge uniformément vers cette fonction.
	\end{remark}
	\section{Polynôme de Lagrange}
	Étant donné une fonction continue $f$ sur un segment $[a,b]$ et $n$ évaluations connues de $f$ $f(x_1),\ldots, f(x_n)$ on peut construire le polynôme $p_n$ de degré $n$ passant par ces $n$ évaluations.
\end{document}
