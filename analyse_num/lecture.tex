\documentclass[12pt]{book}
\usepackage[top=1in, bottom=1in, left=1in, right=1in]{geometry}
\usepackage[onehalfspacing]{setspace}
\usepackage[french]{babel}
              
\usepackage{amsmath, amssymb, amsthm}
\usepackage{mathrsfs}
\usepackage{amsfonts}
\usepackage{mathtools}
\usepackage{stmaryrd}

\newcommand{\defeq}{\vcentcolon=}
\newcommand{\eqdef}{=\vcentcolon}
                   
\usepackage{enumerate, enumitem}
\usepackage{fancyhdr, graphicx, proof, comment, multicol}
\usepackage[none]{hyphenat}
\usepackage{dirtytalk}
\usepackage{proof}
           
\usepackage{graphicx}
\usepackage{tikz-cd}
            
\newtheorem{lemma}{Lemme}[section]
\newtheorem{theorem}[lemma]{Théorème}
\newtheorem{cor}[lemma]{Corollaire}
\newtheorem{prop}[lemma]{Proposition}
                    
\theoremstyle{definition}
\newtheorem{definition}[lemma]{Définition}
\newtheorem{example}[lemma]{Exemple}
\newenvironment{comments}{}{}
\newtheorem{notation}{Notation}

\theoremstyle{remark}
\newtheorem*{remark}{Remarque}

\begin{document}
	\chapter{La représentation des nombres}
	\section{Introduction}
	Nous représentons les nombres avec une notation positionnelle. Dans les ordinateurs nous utilisons la base $\beta = 2$. Chaque nombre représentable doit pouvoir s'écrire comme suite finie de 0 et de 1.
	\begin{notation}
		On note l'ensemble des nombres représentables $\mathcal{F}(\beta, t, L, U)$, chaque  nombre non nul de cet ensemble s'écrit sous la forme \[
			(-1)^s\cdot(0.\alpha_1\ldots\alpha_t)_{\beta}\cdot \beta^{e}
		.\] 
		Cette représentation s'appelle représentation en \emph{virgule flottante}, on appelle e \emph{l'exposant}, s le \emph{signe} et $0 \le \alpha_i < \beta$ sont des entiers qui forment la \emph{mantisse}.
	\end{notation}
	\begin{example}
		On peut considérer $\mathcal{F}(10,3,-2,2)$, dans cet ensemble de nombres on a \[
			-48.3 = (-1)^1\cdot (0.438)\cdot10^2
		.\] 
		Cependant d'autres nombres n'appartiennent pas à cet ensemble comme par exemple \[
			3,141 = (-1)^0\cdot(0.3141)\cdot10^1
		.\] 
	\end{example}
	\section{Représentation des nombres dans l'ordinateur}
	\begin{definition}[64-floating point representation]
		Dans les ordinateurs l'ensemble des nombres représentables est $\mathcal{F}(2,53,-1021,1024)$.
		Il y a un bit de signe, 52 bits pour représenter les $\alpha_i$ avec $\alpha_1 \neq 0$ et 11 bits pour l'exposant, 1 pour le signe et 10 pour sa valeur. Cette représentation s'appelle \emph{64-floating point representation}.
	\end{definition}
	Nous allons étudier la structure de cet ensemble. La plus petite valeur en valeur absolue est \[
		x_{\min} = (0.100\ldots0)\cdot 2^L \sim 2\cdot 10^{-308}
	\]  
	et la plus grande est \[
		x_{\max} = (0.11\ldots1)\cdot2^U \sim 1.8\cdot 10^{308}
	.\] 
	\begin{remark}
		Tous les nombres dans $\mathcal{F}$ sont de la forme $\frac{p}{2^n}, \; n \in \mathbf{N}$ dans un ensemble borné. On en déduit que les rationnels n'appartiennent pas tous à $\mathcal{F}$. Le fait que cet ensemble soit de plus discret justifie la définition suivante.
	\end{remark}
	\begin{definition}[Spacing]
		On appelle \emph{spacing} la distance entre deux nombres consécutifs dans $\mathcal{F}$.
	\end{definition}
	Pour un exposant $p$, le nombre le plus proche de  $\beta^p$ est à une distance  $\beta^{p+1-t}$. Il est important de noter que la répartition est uniforme sur l'intervalle $[\beta^p,\beta^{p+1}]$ mais la distance entre chaque nombre de $\mathcal{F}$ dépend de $p$.
	\begin{remark}
		On peut se demander pour quel $p$ on a  $\beta^{p+1-t} = 1$, c'est le cas pour $p=52$ et donc dans l'intervalle $[2^{52},2^{53}]$ seuls les entiers sont représentés. 
	\end{remark}
	\subsection{Approximation de $\mathbf{R}$ dans $\mathcal{F}(2,53,L,U)$}
	La mantisse d'un nombre réel est \textit{a priori} infinie. On pose alors une fonction d'approximation $fl : \mathbf{R} \longmapsto \mathcal{F}$ qui représente $x\in \mathbf{R}$ avec la représentation floating point.\\
	On s'intéresse alors à estimer la valeur de la différence $\|x-fl(x)\|$. Pour $x\in [\beta^{e-1},\beta^e]$ on a 
	 \begin{align*}
		 \|x-fl(x)\| &\le \frac{1}{2} \cdot \text{spacing} \\
			   &\le \frac{1}{2}\beta^{e-t}
	.\end{align*}
	On peut également regarder l'erreur relative \[
		\frac{\|x-fl(x)\|}{x} \le \frac{1}{2}\beta^{e-t}\cdot\beta^{1-e} = 2^{-53} \sim 10^{-16}
	.\] 
	\begin{definition}[Machine precision]
	On appelle la valeur $2^{53} \sim 10^{-16}$ \emph{machine precision} et on la note $u$.
	\end{definition}

	\begin{theorem}
		Soit $x \in \mathbf{R}$ alors $\exists fl(x) \in \mathcal{F}(2,53,L,U)$ tel que \[
			fl(x) = x(1+\varepsilon), \qquad \|\varepsilon\| \le
		.\] 
	\end{theorem}
	\section{Opérations dans $\mathcal{F}$}
	Il est important de noter que $\mathcal{F}$ n'est pas muni d'une structure de corps, il n'est donc pas nécessairement stable par addition. Ainsi pour $x,y \in \mathbf{R}$
	\begin{align*}
		x+ y \longmapsto fl(x) + fl(y)	
	\end{align*} peut ne pas appartenir à $\mathcal{F}$.
	Ainsi $x+y$ est représenté par  $fl(fl(x)+fl(y))$.
	On veut contrôler la valeur C telle que \[
		\frac{\|fl(fl(x)+fl(y)) -(x+y)\|}{\|x+y\|} \le C\cdot u
	.\] 
	En général on peut essayer de définir la stabilité d'un problème.

	\begin{definition}[Stabilité d'un problème]
		La résolution du problème $y = G(x)$ est \emph{stable} si à petite perturbation $\delta_x$ de x correspond une petite perturbation $\delta_y$ de y. 
	\end{definition}
	\begin{definition}[Conditionnement]
		On appelle \emph{conditionnement absolu} du problème la valeur \[
			K_{abs} \defeq \sup_{\delta_x}\frac{\|\delta_y\|}{\delta_x}
		.\] 	
		On appelle \emph{conditionnement relatif} du problème la valeur \[
			K_{rel} \defeq \sup_{\delta_x} \frac{\|\delta_y\|/\|y\|}{\|\delta_x\|/\|x\|}
		.\] Cette dernière valeur mesure la stabilité du système. 
	\end{definition}
	\subsection{L'arithmétique finie}
	Nous allons appliquer ces concepts de base à l'arithmétique finie.
	\begin{align*}
		\frac{\|fl(fl(x_1)+fl(x_2))-(x_1+x_2)\|}{\|x_1+x_2\|} &= \frac{\|((x_1+x_2)(1+\varepsilon))(1+\varepsilon) - (x_1+x_2)\|}{\|x_1+x_2\|} \\
								      &\le \max_{x_1,x_2}(\frac{\|x_1\|}{\|x_1+x_2\|} + \frac{\|x_2\|}{\|x_1+x_2\|} +1)\cdot u \\
	.\end{align*}
	En conclusion si $x_1,x_2$ sont du même signe le conditionnement relatif est faible puisque inférieur à 3, lorsque $x_1 \sim~-x_2$ l'opération est instable. 
\end{document}
