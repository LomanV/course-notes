\documentclass[12pt]{article}
\usepackage[top=1in, bottom=1in, left=1in, right=1in]{geometry}

\usepackage[onehalfspacing]{setspace}

\usepackage{amsmath, amssymb, amsthm}
\usepackage{enumerate, enumitem}
\usepackage{fancyhdr, graphicx, proof, comment, multicol}
\usepackage[none]{hyphenat}
\usepackage{dirtytalk}
\binoppenalty=\maxdimen
\relpenalty=\maxdimen

\usepackage{microtype}
\usepackage{mathpazo}
\usepackage{mdframed}
\usepackage{parskip}
\linespread{1.1}
\usepackage{graphicx}
\usepackage{subfig}

\usepackage{mathrsfs}
\usepackage{amsfonts}
\usepackage{amsmath}
\usepackage{amssymb}

\usepackage{mathtools}
\newcommand{\defeq}{\vcentcolon=}
\newcommand{\eqdef}{=\vcentcolon}

\newenvironment{statement}[1]
{\begin{mdframed}[linewidth=0.6pt]
        \textsc{Statement #1:}

}
    {\end{mdframed}}

\newcommand{\R}{\mathbf{R}}
\newcommand{\C}{\mathbf{C}}
\newcommand{\Z}{\mathbf{Z}}
\newcommand{\N}{\mathbf{N}}
\newcommand{\Q}{\mathbf{Q}}
\newcommand{\T}{\mathbf{T}}

\newcommand{\de}{\mathrm{d}}

\begin{document}
        \noindent
\textbf{Systèmes dynamiques} \hfill \textbf{Vezin Lomàn}\\
\normalsize MAT551  \hfill Date de rendu: 29/09/2020\\

\begin{center}
\textbf{Exercice à rendre 1}
\end{center}
        
Soit $\alpha \not\in \Q$ et $f$ la dynamique du tore $\T^{2}$ définie par \[
        f(x, y) = f(x+\alpha, x + y)
.\]  
On vérifie dans un premier temps que la mesure de Lebesgue sur le tore est $f-$invariante. On commence par remarquer que $f$ est une bijection du tore dans lui même d'inverse  \[
        f^{-1}(x, y) = (x - \alpha, y - x + \alpha)
        .\] 
        Aussi par la formule du changement de variable, en remarquant que $|\det d_{f}| = 1$ on obtient pour tout borélien $\mathcal{A}$ \[
        \mu(f^{-1}(\mathcal{A})) = \int_{f^{-1}(\mathcal{A})}1 \de\mu = \int_{\mathcal{A}} |\det d_{f}|\de\mu = \int_{\mathcal{A}}1\de\mu = \mu(\mathcal{A}).\] Ce qui donne le résultat voulu.

        \bigskip

        On montre à présent que le système probabiliste $(\T^{2}, \mathcal{B}, f, \mu)$ est ergodique. \\
        Pour cela on applique un corollaire du cours et on vérifie que l'espace propre de l'opérateur de Koopman associé à la valeur propre 1 est de dimension 1, c'est à dire que les seules fonctions $L^{2}(Leb)$ invariantes sont constantes. On considère pour cela la base hermitienne $(e_{k})_{k \in \Z^{2}}$ de $L^{2}$ définie par $e_{k}(x) \defeq e^{\langle k, x \rangle}$. \\

        Soit donc $\phi \in L^{2}$ satisfaisant $\phi \circ f = \phi$ et soit $\sum_{k_1, k_2} a_{(k_1, k_2)}e_{(k_1, k_2)}$ sa série de Fourier. On regarde les coefficients de Fourier de $\phi \circ f$.
        
        \begin{align*}
                a_{(k_1, k_2)}(\phi \circ f) &= \int_{\T^{2}}\phi(f(x_{1}, x_{2}))e^{-2i\pi \langle (k_1, k_2), (x_1, x_2) \rangle} \de Leb \\
                \text{Par la formule de} & \text{ changement de variable et le fait que } |\det d_{f^{-1}}| = 1 \\
                                         &= \int_{\T^{2}} \phi(x_1, x_2)e^{-2i\pi \langle(k_1, k_2), f^{-1}(x_1, x_2) \rangle} \de Leb \\
                                         &= \int_{\T^{2}} \phi(x_1, x_2)e^{-2i\pi \langle(k_1, k_2), (x_1 - \alpha, x_2 - x_1 + \alpha) \rangle} \de Leb \\
                                 &= e^{2i\pi\alpha(k_1-k_2)} \int_{\T^{2}} \phi(x_1, x_2)e^{-2i\pi \langle (k_1 - k_2, k_2), (x_1, x_2) \rangle} \de Leb \\
                                 &= e^{2i\pi\alpha(k_1-k_2)} a_{(k_1-k_2, k_2)}(\phi) 
        .\end{align*}

        La relation $\phi = \phi \circ f$ garantit alors  
        \begin{equation}
                a_{(k_1, k_2)}(\phi) = e^{2i\pi\alpha(k_1-k_2)} a_{(k_1-k_2, k_2)}(\phi).
        \end{equation}

        Ce qui donne en passant au module au carré
        \begin{equation}
                |a_{k_1, k_2}|^2 = |a_{k_1 - nk_2, k_2}|^2 \qquad \forall n \in \Z^{*}.
        \end{equation}
        Puisque par hypothèse $\phi$ est $L^{2}$ la série des $|a_{(k_1, k_2)}|^{2}(\phi)$ converge, la relation (2) impose \[
                a_{(k_1, k_2)}(\phi) = 0 \text{ si  }  k_2 \neq 0
        \] puisque sinon la série comprendrait un nombre infini de termes égaux et strictement positifs, une contradiction. \\

        Finalement comme $\alpha$ est irrationnel $e^{2ik_1\pi\alpha} \neq 1 \quad \forall k_1 \in \Z^{*}$, ce qui garantit par la relation (1) que \[
        a_{(k_1, k_2)}(\phi) = 0 \text{ si  }  k_1 \neq 0
        .\]  Puisque seul le coefficient $a_{0, 0}$ peut être non nul $\phi$ est une constante. 
        \bigskip

        On montre finalement que le système n'est pas mélangeant. Par le cours on sait que si le système est mélangeant alors 1 est l'unique valeur propre de l'opérateur de Koopman $U_{f}$. Il suffit alors de trouver une valeur propre différente de 1 pour notre opérateur. Définissons une application mesurable et $L^{2}$ sur le tore 
        \begin{align*}
                \phi : \T^{2} &\longmapsto \T^{2} \\
                (x, y) &\longmapsto (\frac{x}{\alpha}, 0)
        .\end{align*}

        On remarque que le passage au quotient sur le tore nous permet d'écrire \[
                \phi(x, y) = (\frac{x}{\alpha}, 0) = (\frac{x}{\alpha} + 1, 0) = \frac{1}{\alpha}(x + \alpha, 0) = \frac{1}{\alpha}(\phi \circ f)(x, y)
        .\] 
        Soit donc \[
                U_{f}(\phi) = \alpha\phi
        .\]
        $\alpha$ est donc valeur propre de $U_{f}$ et comme $\alpha \not\in \Q, \alpha \neq 1$. Cela montre le résultat et conclut l'exercice. 
\end{document}
