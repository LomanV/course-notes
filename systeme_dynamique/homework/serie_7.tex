\documentclass[12pt]{article}
\usepackage[top=1in, bottom=1in, left=1in, right=1in]{geometry}

\usepackage[onehalfspacing]{setspace}

\usepackage{amsmath, amssymb, amsthm}
\usepackage{enumerate, enumitem}
\usepackage{fancyhdr, graphicx, proof, comment, multicol}
\usepackage[none]{hyphenat}
\usepackage{dirtytalk}
\binoppenalty=\maxdimen
\relpenalty=\maxdimen

\usepackage{microtype}
\usepackage{mathpazo}
\usepackage{mdframed}
\usepackage{parskip}
\linespread{1.1}
\usepackage{graphicx}
\usepackage{subfig}

\usepackage{mathrsfs}
\usepackage{amsfonts}
\usepackage{amsmath}
\usepackage{amssymb}

\usepackage{mathtools}
\newcommand{\defeq}{\vcentcolon=}
\newcommand{\eqdef}{=\vcentcolon}

\newenvironment{statement}[1]
{\begin{mdframed}[linewidth=0.6pt]
        \textsc{Statement #1:}

}

\newenvironment{ex}[1]
{\begin{mdframed}[linewidth=0.6pt]
        \textsc{Exercice #1:}

}
    {\end{mdframed}}

\newcommand{\R}{\mathbf{R}}
\newcommand{\C}{\mathbf{C}}
\newcommand{\Z}{\mathbf{Z}}
\newcommand{\N}{\mathbf{N}}
\newcommand{\Q}{\mathbf{Q}}

\newcommand{\de}{\mathrm{d}}

\begin{document}
        \noindent
\textbf{Systèmes Dynamiques} \hfill \textbf{Vezin Lomàn}\\
\normalsize MAT551 \hfill Date de rendu: 23/11/2020\\

\begin{center}
\textbf{Exercices à rendre}
\end{center}

\begin{ex}{1}
       \begin{enumerate}
               \item Le rayon de convergence de la série entière $u$ est $\frac{1}{\rho(A)}$.
               \item On a $e^{u(z)} = \frac{1}{\det(I_{d}-zA)}$.
       \end{enumerate} 
\end{ex}

1. Par le cours on sait que \[
        \#Per_{n}(\Sigma_{A},\sigma) = tr(A^{n})
.\] On applique le critère de Cauchy pour trouver le rayon de convergence $R$ de la série entière $u$.

Il est facile de voir que  \[
        \lim_{n} \frac{1}{\sqrt[n]{n}} = 1
.\] On cherche alors à calculer la limite suivante \[
\limsup_{n} \sqrt[n]{|tr(A^{n})|} 
.\] Puisque la trace d'une matrice est la somme de ses valeurs propres et que les valeurs propres de $A^{n}$ sont les puissances $n-$ème des valeurs propres de $A$ on a asymptotiquement \[
|tr(A^{n})| \sim_{n} \rho(A)^{n}
.\] On en déduit donc \[
\limsup_{n} \sqrt[n]{\frac{\#Per_{n}(\Sigma_{A}, \sigma)}{n}} = \rho(A)
.\]  

Le critère de Cauchy donne donc \[
        R = \frac{1}{\rho(A)}
.\] Il est en particulier infini lorsque toutes les valeurs propres de $A$ sont égales à 0 puisqu'alors le terme général de $u$ vaut 0. 

\bigskip

2. La convergence de $u$ sur $]-R,R[$ nous permet d'écrire sur ce domaine
 \begin{align*}
         e^{u(z)} &= \exp(\sum_{n=1}^{\infty} \frac{\#Per_{n}(\Sigma_{A})}{n}z^{n}) \\
                  &= \prod_{n=1}^{\infty} \exp(\frac{\#Per_{n}(\Sigma_{A})}{n}z^{n}) \\
                  &= \prod_{n=1}^{\infty} \exp(\frac{\mathrm{Tr}((zA)^{n})}{n}) \\
                  &= \prod_{n=1}^{\infty} (\exp(\mathrm{Tr}((zA)^{n}))^{\frac{1}{n}}
.\end{align*}
Et comme pour une matrice $B$ \[
        \exp(\mathrm{Tr}(B)) = \det(\exp(B))
,\] on obtient \[
e^{u(z)} = \prod_{n=1}^{\infty} (\det(\exp((zA)^{n}))^{\frac{1}{n}} 
.\]  

On remarque une expression très semblable à celle recherchée mais je ne parviens pas à conclure.

\newpage
        
\begin{ex}{2}
        \begin{enumerate}
                \item La période $p(i)$ d'une matrice irréductible $A \in \mathcal{M}_{d}(\{0,1\})$ est indépendante de $i \in \{1,\ldots,d\}$.
                \item Une matrice irréductible est primitive si et seulement si sa période vaut 1.
                \item Le système $(\Sigma_{A},\sigma)$ est topologiquement mélangeant si et seulement si $A$ est primitive.
        \end{enumerate}
\end{ex}

1. Soit $A$ une telle matrice supposée irréductible, soient $i,j \in \{1,\ldots,d\}$. Comme $A$ est irréductible il existe par définition des entiers $n, m > 0$ tels que  \[
        (A^{m})_{ij} > 0 \quad \text{et} \quad (A^{n})_{ji} > 0
.\] 
Soit de plus un entier $s > 0$ tel que \[
        (A^{s})_{jj} > 0
,\] son existence est encore garantie par irréductibilité de $A$. 

Remarquons que $(A^{2s})_{jj} > 0$, en effet
\begin{align*}
        (A^{2s})_{jj} &= \sum_{k=1}^{d} (A^{s})_{jk}(A^{s})_{kj} \\        
                      &\ge (A^{s})_{jj}(A^{s})_{jj} > 0
,\end{align*} puisque les coefficients de $A$ sont dans $\{0,1\}$.

En utilisant cette majoration et l'expression du produit matriciel deux fois on obtient par la suite
\begin{align*}
        (A^{m+s+n})_{ii} &= \sum_{k=1}^{d} (A^{m+s})_{ik}(A^{n})_{ki} \\
                         &\ge (A^{m+s})_{ij}(A^{n})_{ji} \\
                         &\ge \sum_{k=1}^{d} (A^){m})_{ik}(A^{s})_{kj}(A^{n})_{ji} \\
                         &\ge (A^{m})_{ij}(A^{s})_{jj}(A^{n})_{ji} > 0
,\end{align*} par hypothèse sur $n,m,s$.

On en déduit par définition de $p(i)$ que $p(i) \;|\; m+n+s$ et par notre première observation $p(i) \;|\; m+n+2s$ ainsi,  \[
        p(i) \;|\; s = m+2s+n - (m+s+n)
.\]  

Comme $s$ est arbitraire avec cette propriété et comme $p(j)$ est le pgcd de tels $s$ on doit avoir  \[
        p(i) \le p(j)
.\] Le problème étant symétrique on peut montrer par le même raisonnement $p(j) \le p(i)$ et donc \[
p(i) = p(j)
.\] Puisque le choix de $i,j \in \{1,\ldots,d\}$ était arbitraire nous pouvons conclure. 

\bigskip

2. Supposons $A$ primitive. Soit donc un entier $l > 0$ tel que \[
        (A^{l})_{ij} > 0, \quad \forall i,j \in \{1,\ldots,d\} 
.\] 

Fixons $i \in \{1,\ldots,d\}$, on a en particulier $(A^{l})_{ii} > 0$. Par définition du produit matriciel cela implique que toute colonne de $A$ doit contenir au moins un $1$. On en déduit alors
\[
        (A^{l+1})_{ii} = \sum_{k=1}^{d} (A^{l})_{ik} (A)_{ki} > 0
.\] 

Ainsi $p(i)$ divise $l$ et $l+1$, on a donc $p(i) = 1$. Par la question 1. on peut conclure que la période de $A$ vaut 1.

\medskip

Supposons à présent que $A$ soit de période 1. Fixons $i \in \{1,\ldots,d\}$ et montrons l'existence d'un entier $n_{i} > 0$ tel que \[
        (A^{n})_{ii} > 0, \quad \forall n \ge n_{i}
.\]  

On pose $P \defeq \{k \in \N \;|\; (A^{k})_{ii} > 0\} \subset \N$. Remarquons d'abord que $P$ est stable par addition. Si $m,n \in P$ alors
\begin{align*}
        (A^{n+m})_{ii} &= \sum_{k=1}^{d} (A^{n})_{ik}(A^{m})_{ki} \\
                       &\ge (A^{n})_{ii}(A^{m})_{ii} > 0
,\end{align*} comme les coefficients de $A$ sont dans  $\{0,1\}$.


Comme la période de $A$ est 1 par hypothèse, $1 \in P$ et donc le pgcd des éléments de $P$ est forcément 1. Par les propriétés du pgcd on a l'existence de $l > 0, m_{j} \in \Z, p_{j} \in P$ tels que \[
\sum_{j=1}^{l} m_{j}p_{j} = 1
.\] 

En séparant les termes positifs dont on nomme la somme $P^{+}$ et les termes négatifs dont on nomme la somme $P^{-}$ on obtient \[
1 = P^{+} - P^{-}, \quad P^{+}, P^{-} \ge 0
,\] la stabilité par addition garantit de plus $P^{+}, P^{-} \in P$. 

Posons maintenant $n_{i} = P^{-}(P^{-}-1)$ et prenons $n \ge n_{i}$ alors par division euclidienne on peut écrire \[
n = qP^{-} + r, \quad 0 \le r < P^{-}, \; q \ge 0
.\] On en déduit par définition de $n_{i}$ que $q > P^{-} - 1$ et donc $q-r > 0$.

Finalement comme $P^{+}-P^{-} = 1$
\begin{align*}
        n &= qP^{-} + r \\
          &= qP^{-} +rP^{+} - rP^{-} \\
          &= (q-r)P^{-} + rP^{+} \in P
\end{align*} toujours par stabilité de $P$ pour l'addition. Par définition de $P$ nous avons bien trouvé un entier $n_{i} > 0$ avec la propriété souhaitée.

\medskip

Soit $n_{i}$ avec la propriété souhaitée, fixons $j \in \{1,\ldots,d\}$. Comme $A$ est irréductible il existe un entier $l_{j} > 0$ tel que $(A^{l_{j}})_{ij} > 0$. Ainsi pour $n \ge n_{i} + l_{j}$ on a
\begin{align*}
        (A^{n})_{ij} &= \sum_{k=1}^{d} (A^{n-l_{j}})_{ik}(A^{l_{j}})_{ki} \\
         &\ge (A^{n-l_{j}})_{ii}(A^{l})_{ij} > 0
.\end{align*} En posant \[
N \defeq \sup_{i,j} n_{i}l_{j}
,\] bien défini comme $\{1,\ldots,d\}$ est un ensemble fini on obtient par construction \[
(A^{N})_{ij} > 0, \quad \forall i,j \in \{1,\ldots,d\}  
\] et $A$ est primitive.

\bigskip

3. Supposons tout d'abord $(\Sigma_{A}, \sigma)$ topologiquement mélangeant. Fixons $i,j \in \{1,\ldots,d\}$. Nous savons que la restriction des cylindres $[i], [j]$ dans $\Sigma_{A}$ sont des ouverts. L'hypothèse de mélange nous donne donc l'existence d'un $N_{ij}$ tel que
\[
        \sigma^{n}[i] \cap [j] \neq \emptyset, \quad \forall n \ge N_{ij}
,\] c'est à dire l'existence d'une suite $u$ de $\Sigma_{A}$ telle que \[
u_{0} = i \; \text{et} \; \sigma^{n}(u)_{0} = u_{n} = j
\] et donc par définition de la matrice d'adjacence, \[
(A^{n})_{ij} > 0
.\] En prenant \[
\tilde{N} \defeq \sup_{i,j} N_{ij}
\] qui est bien défini puisque $\{1,\ldots,d\}$ est fini, on a bien \[
(A^{\tilde{N}})_{xy} > 0 \quad \forall x,y \in \{1,\ldots,d\} 
.\] $A$ est primitive.

\medskip

Supposons réciproquement que $A$ soit primitive. Soit $N > 0$ un entier tel que  \[
        (A^{N})_{ij} > 0, \quad \forall i,j \in \{1,\ldots,d\} 
.\] Comme mentionné précédemment cela implique par définition du produit matriciel que chaque colonne de $A$ doit contenir au moins un 1. On en déduit
\begin{align*}
        (A^{N+1})_{ij} &= \sum_{k=1}^{d} (A^{N})_{ik}(A)_{kj} > 0, \quad \forall i,j \in \{1,\ldots,d\} 
.\end{align*} Une récurrence directe donne alors \[
(A^{n})_{ij} > 0, \; \forall n \ge N, \quad \forall i,j \in \{1,\ldots,d\} 
.\] 

Soient $\mathcal{U}, \mathcal{V} \subset \Sigma_{A}$ deux ouverts non vides. Ces derniers doivent en particulier contenir une boule ouverte de diamètre strictement positif. Puisque les cylindres forment un recouvrement ouvert de $\Sigma_{A}$ dont le diamètre tend vers 0, on peut en intersectant $\mathcal{U}, \mathcal{V}$ avec des cylindres de diamètre suffisamment petit se ramener à l'étude de ces derniers. Quitte à réduire encore la taille d'un des deux ouverts nous pouvons supposer ces cylindres de longueur égale $k$. Notons ces deux cylindres $ \mathcal{C}_{1} \defeq [i_{1}\ldots i_{k}], \;\mathcal{C}_{2} \defeq [j_{1}\ldots j_{k}]$ et supposons par l'absurde qu'il existe $n \ge N$ tel que $\sigma^{n}\mathcal{C}_{1} \cap \mathcal{C}_{2} = \emptyset$. Cela signifie qu'il n'existe pas dans $\Sigma_{A}$ de suite $u$ telle que  \[
        u_{0} = i_{1} \; \text{et} \; u_{n} = \sigma^{n}(u)_{0} = j_{1}
\] puisque sinon on aurait \[
\sigma^{n}(u) = j_{1}\ldots j_{k}\ldots \; \in \mathcal{C}_{2} 
\] pour une suite $u \in \Sigma_{A}$ bien choisie comme le mot $j_{1}\ldots j_{k}$ est réalisable dans $\Sigma_{A}$. Mais par définition de la matrice d'adjacence cela signifie pour ce $n$ que  \[
(A^{n})_{i_{1}j_{1}} = 0
,\] une contradiction avec notre observation précédente. Le système $(\Sigma_{A},\sigma)$ est donc nécessairement mélangeant.
\end{document}
