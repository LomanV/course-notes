\documentclass[12pt]{article}
\usepackage[top=1in, bottom=1in, left=1in, right=1in]{geometry}
\usepackage[french]{babel}

\usepackage[onehalfspacing]{setspace}

\usepackage{amsmath, amssymb, amsthm}
\usepackage{enumerate, enumitem}
\usepackage{fancyhdr, graphicx, proof, comment, multicol}
\usepackage[none]{hyphenat}
\usepackage{dirtytalk}
\binoppenalty=\maxdimen
\relpenalty=\maxdimen

\usepackage{microtype}
\usepackage{mathpazo}
\usepackage{mdframed}
\usepackage{parskip}
\linespread{1.1}
\usepackage{graphicx}
\usepackage{subfig}

\usepackage{mathrsfs}
\usepackage{amsfonts}
\usepackage{amsmath}
\usepackage{amssymb}

\usepackage{mathtools}
\newcommand{\defeq}{\vcentcolon=}
\newcommand{\eqdef}{=\vcentcolon}

\newenvironment{statement}[1]
{\begin{mdframed}[linewidth=0.6pt]
        Exercice \textsc{#1:}

}
    {\end{mdframed}}

\newcommand{\R}{\mathbf{R}}
\newcommand{\C}{\mathbf{C}}
\newcommand{\Z}{\mathbf{Z}}
\newcommand{\N}{\mathbf{N}}
\newcommand{\Q}{\mathbf{Q}}

\newcommand{\de}{\mathrm{d}}

\begin{document}
        \noindent
\textbf{Systèmes Dynamiques} \hfill \textbf{Vezin Lomàn}\\
\normalsize MAT551 \hfill Due Date: 06/10/2020\\

\begin{center}
\textbf{Devoir maison 2}
\end{center}

\begin{statement}{1}
        On considère le système dynamique $(\mathbf{T}^{1}, \mathcal{B}, f, Leb)$ où $f$ est le doublement de l'angle. Pour $\phi : \mathbf{T}^{1} \longmapsto \R$ continue avec $\phi(0) \neq \frac{1}{2}(\phi(\frac{1}{3})+\phi(\frac{2}{3}))$, la moyenne temporelle $\frac{1}{n}\sum_{k=0}^{n-1} U_{f}^{k}\phi$ ne converge pas dans $L^{\infty}(Leb)$.
\end{statement}

\begin{proof}
        Soit $f$ le doublement de l'angle donné par 
         \begin{align*}
                 f : \mathbf{T}^{1} &\longmapsto \mathbf{T}^{1} \\
                 x &\longmapsto 2x
        ,\end{align*} $f$ est en particulier continue sur le cercle. Ainsi la $n-$ème somme \[
        U_{n, f} \defeq \frac{1}{n}\sum_{k=0}^{n-1} U_{f}^{k}\phi 
        \] est continue sur le cercle comme somme de compositions de fonctions continues. \\  

        On montre que le système étudié est ergodique. On considère le $\pi$-système des intervalles ouverts de $\mathbf{T}^{1}$. Soit $(a, b)$ un tel ouvert, on a d'une part $Leb(a,b) = b-a$ et d'autre part \[
                Leb(f^{-1}(a,b)) = Leb((\frac{a}{2}, \frac{b}{2})\cup (\frac{a}{2}+\frac{1}{2}, \frac{b}{2}+\frac{1}{2})) = b-a
        \] comme ces deux ensembles sont disjoints. La mesure de Lebesgue sur $\mathbf{T}^{1}$ est donc $f$ invariante. De plus pour $\phi : \mathbf{T}^{1} \longmapsto \R$ mesurable satisfaisant $\phi\circ f = \phi, \; Leb \; p.p.$ on a \[
        \phi(x) = \phi(2x)\; \text{pour presque tout} \; x \in \mathbf{T}^{1},
\] soit donc pour $n \in N$ quelconque \[
\phi[0, \frac{1}{2^{n}}) = \phi[0, \frac{1}{2^{n+1}})
.\] On en déduit que $\phi$ est constante pour presque tout $x \in \mathbf{T}^{1}$. Le système est donc ergodique. \\

Comme $\phi$ est continue sur $\mathbf{T}^{1}$ compact elle est bornée donc en particulier $L^{\infty}$ et  \[
\int_{\mathbf{T}^{1}}|\phi|^{2}\de Leb \le \|\phi\|_{L^{\infty}}^{2}\int_{\mathbf{T}^{1}}1\de Leb < \infty
.\] On en déduit que $\phi$ est $L^{2}$. Par le même raisonnement on obtient directement que $\phi$ est $L^{1}$. \\

Par le théorème ergodique en moyenne on a alors dans $L^{2}$ \[
        U_{n,f}\phi \underset{n\to \infty}{\to} E[\phi|\mathcal{J}] = \int_{\mathbf{T}^{1}}\phi \de Leb
.\] Ainsi $U_{n,f}\phi$ converge dans $L^{2}$ vers une constante $c \in \R$. \\ 

Supposons par l'absurde que $U_{n,f}\phi$ converge également dans $L^{\infty}$. La limite uniforme d'une fonction continue est continue, ainsi $U_{n,f}\phi$ converge vers une fonction continue $g$. De plus  \[
\int_{\mathbf{T}^{1}}|U_{n,f}\phi - g|^{2} \de Leb \le \|U_{n,f}\phi - g\|_{L^{\infty}}^{2}\int_{\mathbf{T}^{1}}1\de Leb \underset{n \to \infty}{\to} 0
.\]  
Ainsi $U_{n,f}\phi \to g$ dans $L^{2}$ et donc $g$ est constante égale à $c$ par unicité de la limite. \\

Remarquons que $U_{n,f}\phi(0) = \phi(0)$ comme 0 est un point fixe de $f$. De plus comme $f(\frac{1}{3}) = \frac{2}{3}$ et $f(\frac{2}{3}) = \frac{1}{3}$ on obtient \[
        U_{n, f}\phi(\frac{1}{3}) = \frac{1}{n}\sum_{k}^{n-1}U_{f}^{k}\phi(\frac{1}{3}) = \begin{cases}
                \frac{1}{2}(\phi(\frac{1}{3})+\phi(\frac{2}{3})) \; \text{si} \; n \; \text{est pair} \\
                \frac{n-1}{2n}(\phi(\frac{1}{3})+\phi(\frac{2}{3})) + \frac{1}{n}\phi(\frac{1}{3}) \; \text{sinon}
        \end{cases}
.\]  

Ceci implique par passage à la limite en se souvenant que $g$ est continue partout \[
        g(0) = \phi(0) = \frac{1}{2}(\phi(\frac{1}{3})+\phi(\frac{2}{3})) = g(\frac{1}{3}) 
,\] une contradiction.  

\end{proof}

\newpage

\begin{statement}{2}
       On étudie la capacité orbitale d'un borélien $\mathcal{A}$ d'un espace métrisable compact $X$ pour une application continue $T : X \longmapsto X$.
\end{statement}

\textbf{1.} Dans un premier temps on montre que la limite \[
\lim_{n\to \infty}\frac{1}{n}\sup_{x\in X}\#\{0\le k < n, \; T^{k}x \in \mathcal{A}\} 
\] existe et ainsi que la capacité orbitale de $\mathcal{A}$ est bien définie. On se propose pour cela d'appliquer le lemme de Fekete, il faut donc montrer qu'il existe un rang $N \in \N$ à partir duquel la suite $(u_{n})_{n}$ définie par \[
u_{n} \defeq \frac{1}{n}\sup_{x\in X}\#\{0\le k < n, \; T^{k}x \in \mathcal{A}\}, \; n \in \N 
\] devient sous additive. \\

Si pour aucun $N \in \N$ on a $u_{N} > 0$ le résultat est évident. Remarquons qu'en vertu du théorème de récurrence de Poincaré cela n'est possible que si $\mu(A) = 0$. \\
S'il existe en revanche un $N \in \N$ tel que $u_{N} > 0$ prenons $N$ minimal avec cette propriété et pour $n, m \ge N$ avec sans perte de généralité $m \ge n$ on a
\begin{align*}
        u_{n+m} &= \frac{1}{m+n}\sup_{x\in X}\#\{0 \le k < n + m, \; T^{k}x \in \mathcal{A}\} \\
                &\le \frac{1}{m+n}\sup_{x\in X}\#\{0 \le k < n, \; T^{k}x \in \mathcal{A}\} + \frac{1}{m+n}\sup_{x\in X}\#\{n\le k < n+m, \; T^{k}x \in \mathcal{A}\} \\
                &\le \frac{n}{m+n}u_{n} + \frac{1}{m+n}\sup_{x\in X}\#\{n\le k < n+m, \; T^{k}x \in \mathcal{A}\}
.\end{align*}
On procède par récurrence sur $m \ge n$ pour conclure.

\bigskip

\textbf{2.} On considère le système probabiliste $(X, \mathcal{B}, T, \mu)$. Grâce aux hypothèses on peut définir une mesure de probabilité et $\mathbf{1}_{\mathcal{A}}$ est intégrable. Supposons $\mu$ $T$-invariante ergodique, on a alors par le théorème ergodique ponctuel, en utilisant les notations de l'exercice 1 \[
        U_{n, T}\mathbf{1}_{\mathcal{A}} \underset{n \to \infty}{\to} E[\mathbf{1}_{\mathcal{A}}|\mathcal{J}] = \int_{X}\mathbf{1}_{\mathcal{A}} \de\mu = \mu(A)
.\] Aussi on peut remarquer que \[
 U_{n, T}\mathbf{1}_{\mathcal{A}}(x) = \frac{1}{n}\#\{0\le k < n, \; T^{k}x \in \mathcal{A}\}
.\] 

On en conclut que \[
        \mathrm{ocap}(\mathcal{A}) \defeq \lim_{n\to \infty} \frac{1}{n}\sup_{x \in X}\#\{0\le k < n, \; T^{k}x \in \mathcal{A}\} \ge \mu(\mathcal{A})
.\] 
\end{document}
