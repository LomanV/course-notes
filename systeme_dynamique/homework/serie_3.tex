\documentclass[12pt]{article}
\usepackage[top=1in, bottom=1in, left=1in, right=1in]{geometry}
\usepackage[french]{babel}

\usepackage{comment}

\usepackage[onehalfspacing]{setspace}

\usepackage{amsmath, amssymb, amsthm}
\usepackage{enumerate, enumitem}
\usepackage{fancyhdr, graphicx, proof, comment, multicol}
\usepackage[none]{hyphenat}
\usepackage{dirtytalk}
\binoppenalty=\maxdimen
\relpenalty=\maxdimen

\usepackage{microtype}
\usepackage{mathpazo}
\usepackage{mdframed}
\usepackage{parskip}
\linespread{1.1}
\usepackage{graphicx}
\usepackage{subfig}

\usepackage{mathrsfs}
\usepackage{amsfonts}
\usepackage{amsmath}
\usepackage{amssymb}

\usepackage{mathtools}
\newcommand{\defeq}{\vcentcolon=}
\newcommand{\eqdef}{=\vcentcolon}

\newenvironment{statement}[1]
{\begin{mdframed}[linewidth=0.6pt]
        \textsc{Statement #1:}

}

\newenvironment{ex}[1]
{\begin{mdframed}[linewidth=0.6pt]
        \textsc{Exercice #1:}

}
    {\end{mdframed}}

\newcommand{\R}{\mathbf{R}}
\newcommand{\C}{\mathbf{C}}
\newcommand{\Z}{\mathbf{Z}}
\newcommand{\N}{\mathbf{N}}
\newcommand{\Q}{\mathbf{Q}}

\newcommand{\de}{\mathrm{d}}

\begin{document}
        \noindent
\textbf{Systèmes dynamiques} \hfill \textbf{Vezin Lomàn}\\
\normalsize MAT551 \hfill Date de rendu: 13/10/2020\\

\begin{center}
\textbf{Devoir maison 3}
\end{center}
        
\begin{ex}{1}
        Pour $\pi : (Y, S) \longmapsto (X,T)$ une semi conjugaison de systèmes topologiques, l'application induite \[
                \pi_{*} : \mathcal{M}(Y, S) \longmapsto \mathcal{M}(X,T)
        \] est surjective.
\end{ex}

\begin{proof}
        L'application induite $\pi_{*}$ est donnée par
        \begin{align*}
                \pi_{*} : \mathcal{M}(Y,S) &\longmapsto \mathcal{M}(X,T) \\
                \mu &\longmapsto \pi_{*}\mu 
        ,\end{align*}
        où $\pi_{*}\mu$ est définie par \[
               \pi_{*}\mu(B) = \mu(\pi^{-1}(B)), \; \forall B \; \text{borélien de } X
        .\] 
        On peut aisément vérifier que $\pi_{*}$ est bien définie. Pour tout ouvert $U \subset Y$ on a par continuité de $\pi$ que $\pi^{-1}(U) \subset X$ est ouvert. De plus comme les propriétés suivantes sont vérifiées pour un ensemble d'indices $I$ dénombrable
        \begin{align*}
                \pi^{-1}(\cup_{i\in I}A_{i}) &= \cup_{i \in I}\pi^{-1}(A_{i}), \\
                \pi^{-1}(\cap_{i\in I}A_{i}) &= \cap_{i\in I}\pi^{-1}(A_{i}), \\
                \pi^{-1}(A_{i})^{c} &= \pi^{-1}(A_{i}^{c})
        ,\end{align*}
la préimage par $\pi$ d'un borélien de $X$ est un borélien de $Y$. Il reste à vérifier que $\pi_{*}\mu$ est bien $T-$invariante. Comme $\pi$ est un morphisme on a que \[
T\circ\pi = \pi\circ S
,\] on en déduit pour tout borélien $A \subset X$ que
\begin{align*}
        T_{*}\pi_{*}\mu(A) &= \mu([T\circ\pi]^{-1}(A)) \\
                           &= \mu([\pi\circ S]^{-1}(A)) \\
                           &= S_{*}\mu(\pi^{-1}(A)) \\
                           &= \mu(\pi^{-1}(A)) \; \; \text{par } S \text{-invariance} \\
                           &= \pi_{*}\mu(A)
.\end{align*}

\medskip

On sait par le cours que $\mathcal{M}_{e}(X,T) = ex(\mathcal{M}(X,T))$, le théorème de Krein Milman nous donne donc \[
        \mathcal{M}(X,T) = \overline{\Delta\mathcal{M}_{e}(X,T)}
        ,\] où $\Delta(A)$ désigne l'enveloppe convexe de $A$. On montrera en dernier lieu l'inclusion 
\begin{equation}
        \mathcal{M}_{e}(X,T) \subset \pi_{*}(\mathcal{M}(Y,S)),
\end{equation}
        qui donne \[
        \mathcal{M}(X,T) = \overline{\Delta\mathcal{M}_{e}(X,T)} \subset \overline{\Delta\pi_{*}(\mathcal{M}(Y,S))}
.\] Cela nous motive à montrer $\overline{\Delta\pi_{*}(\mathcal{M}(Y,S))} = \pi_{*}(\mathcal{M}(Y,S))$.

\medskip

Il est clair que $\pi_{*}(\mathcal{M}(Y,S))$ est convexe par convexité de $\mathcal{M}(Y,S)$ et définition de $\pi_{*}$. Ainsi \[
        \Delta\pi_{*}(\mathcal{M}(Y,S)) = \pi_{*}(\mathcal{M}(Y,S))
.\] 
On montre à présent que l'application induite $\pi_{*}$ est continue. Comme l'espace $\mathcal{M}(Y,S)$ est métrisable on utilise le critère séquentiel. Soit donc $\mu \in \mathcal{M}(Y,S)$ et $(\mu_{n})_{n}$ une suite de mesures $S-$invariantes qui convergent faiblement-* vers $\mu$, c'est à dire  \[
        \forall \phi \in \mathcal{C}(Y), \; \int\phi\de\mu_{n} \to \int\phi\de\mu
.\] 
Soit maintenant $\varphi \in \mathcal{C}(X)$ quelconque, par continuité de $\pi$ la composition est continue, c'est à dire $\varphi \circ \pi \in \mathcal{C}(X)$. De plus, \[
\int\varphi\de\pi_{*}\mu_{n} = \int\varphi\circ\pi\de\mu_{n} \to \int\varphi\pi\de\mu = \int\varphi\de\pi_{*}\mu
.\] Par définition de la convergence faible-* on a $\pi_{*}(\mu_{n}) \rightharpoonup \pi_{*}(\mu)$, $\pi_{*}$ est continue en $\mu$ et donc sur  $\mathcal{M}(Y,S)$ comme le choix de cette dernière était arbitraire.

L'image par une fonction continue d'un compact est un compact, un compact étant fermé dans un espace métrisable on a \[
        \overline{\pi_{*}(\mathcal{M}(Y,S))} = \pi_{*}\mathcal{M}(Y,S)
,\] et on a le résultat.

\medskip

Pour conclure la preuve on montre donc (1). Par surjectivité du morphisme $\pi$ les diracs $\delta_{x}, \; x \in X$ sont toutes dans l'image $\pi_{*}(\mathcal{M}(Y,S))$ comme image des diracs de $Y$. Comme $\pi_{*}(\mathcal{M}(Y,S))$ est convexe il contient toutes les combinaisons linéaires finies de ces diracs, en particulier l'ensemble \[
        \{\frac{1}{n}\sum_{k=0}^{n-1} \delta_{T^{k}x}, \; n \in \N\} 
,\] qui est par le cours dense dans $\mathcal{M}_{e}(X,T)$. Puisque $\pi_{*}(\mathcal{M}(Y,S))$ est compact il doit contenir en particulier l'adhérence \[
\overline{\{\frac{1}{n}\sum_{k=0}^{n-1} \delta_{T^{k}x}, \; n \in \N\}} = \mathcal{M}_{e}(X,T)
.\] Ceci conclut la preuve. 

\newpage

\end{proof}

\begin{ex}{2}
        Avec les notations du devoir de la semaine passée, pour $A \subset X$ un fermé, on a \[
                \sup_{\mu \in \mathcal{M}(X,f)}\mu(A) = \sup_{\mu \in \mathcal{M}_{e}(X,f)}\mu(A) = \mathrm{ocap}(A)
        .\] 
\end{ex}
\begin{proof}
        Soit $A \subset X$ fermé, comme $X$ est compact $A$ l'est aussi. Dans un premier temps, on montre l'égalité \[
        \sup_{\mu \in \mathcal{M}(X,f)}\mu(A) = \sup_{\mu \in \mathcal{M}_{e}(X,f)}\mu(A) 
.\] Comme $\mathcal{M}_{e}(X,f) \subset \mathcal{M}(X,f)$ on a directement \[
\sup_{\mu \in \mathcal{M}(X,f)}\mu(A) \ge \sup_{\mu \in \mathcal{M}_{e}(X,f)}\mu(A) 
.\]  On montre la deuxième inégalité. Soit $d$ la métrique définie sur $X$, $\tilde{d}$ la distance induite entre un point $x$ et un sous ensemble $A$ de $X$ donnée par  \[
\tilde{d}(x,A) \defeq \inf_{y \in A}d(x,y)
.\] On se souvient que cette distance est continue et qu'elle est atteinte pour $A$ compact. Soient de plus $\varepsilon > 0$ et $\phi_{\varepsilon}$ définie par 
\begin{align*}
        \phi_{\varepsilon} : X &\longmapsto \R \\
        x &\longmapsto \begin{cases}
                1 \; \text{si} \; x \in A \\
                1 - \frac{\tilde{d}(x, A)}{\varepsilon} \; \text{si} \; 0 < \tilde{d}(x, A) \le \varepsilon \\
                0 \; \text{sinon}
        \end{cases}
.\end{align*}
$\phi_{\varepsilon}$ est clairement continue. Posons $B_{\varepsilon} \defeq \{x \in X, \; 0 < \tilde{d}(x,A) \le \varepsilon\}$, par continuité à droite de la mesure, pour $\eta$ une mesure
\begin{align*}
        \int |\mathbf{1}_{A}-\phi_{\varepsilon}|\de \eta &\le \int_{B_{\varepsilon}} 1\de \eta \\
                 &\le \eta(B_{\varepsilon}) \underset{\varepsilon \to 0}{\to} 0
.\end{align*}

Soit $\mu \in \mathcal{M}(X,f)$, par le théorème de Choquet il existe une unique décomposition ergodique $M_{\mu}$ telle que
\begin{align*}
        \int\phi_{\varepsilon}\de\mu &= \int(\int\phi_{\varepsilon}\de\nu)\de M_{\mu}(\nu) \\
                                     &\le \sup_{\nu \in \mathcal{M}_{e}(X,f)}\int\phi_{\varepsilon}\de\nu 
.\end{align*}
On en déduit donc 
\begin{align*}
        \mu(A) &\le \int\phi_{\varepsilon}\de\mu + \mu(B_{\varepsilon}) \\
               &\le \sup_{\nu \in \mathcal{M}_{e}(X,f)}\int\phi_{\varepsilon}\de\nu + \mu(B_{\varepsilon})\\
               &\le \sup_{\nu \in \mathcal{M}_{e}(X,f)}\nu(A) + \sup_{\nu \in \mathcal{M}_{e}(X,f)}\int|\mathbf{1}_{A}-\phi_{\varepsilon}|\de\nu + \mu(B_{\varepsilon})\\
               &\le \sup_{\nu \in \mathcal{M}_{e}(X,f)}\nu(A) + \nu(B_{\varepsilon}) + \mu(B_{\varepsilon})
.\end{align*}
On conclut par la remarque précédente que \[
\sup_{\mu \in \mathcal{M}(X,f)}\mu(A) \le \sup_{\mu \in \mathcal{M}_{e}(X,f)}\mu(A) 
,\] en laissant $\varepsilon \to 0$ puis par passage au supremum à gauche. 

\medskip

D'après l'exercice de la semaine précédente, pour toute mesure ergodique $\mu$ on a montré que \[
        \mu(A) \le \mathrm{ocap}(A)
,\] en passant au supremum on obtient une première inégalité \[
\sup_{\mu \in \mathcal{M}_{e}(X,f)} \mu(A) \le \mathrm{ocap}(A)
.\] On montre la seconde inégalité. \\

Pour $N \in \N$ on pose \[
\mathrm{ocap}(A)_{N} \defeq \frac{1}{N}\sup_{x\in X}\#\{0 \le k < N, \; f^{k}x \in A\} 
.\] 
On constate alors que la capacité orbitale est $f-$invariante. En effet 
\begin{align*}
        \mathrm{ocap}(f^{-1}(A))_{N} &= \frac{1}{N}\sup_{x\in X}\#\{0 \le k < N, \; f^{k+1}x \in A\} \\
                                     &\le \frac{1}{N}\sup_{x\in X}\#\{0 \le k < N, \; f^{k}x \in A\} + \frac{2}{N} 
,\end{align*} et de même \[
        \mathrm{ocap}(f^{-1}(A))_{N} \ge \frac{1}{N}\sup_{x\in X}\#\{0 \le k < N, \; f^{k}x \in A\} - \frac{2}{N} \\
.\] On conclut par encadrement et passage à la limite $N \to \infty$. \\

Par définition la capacité orbitale satisfait $0 \le \mathrm{ocap}(A) \le 1$ et cette dernière est \emph{sous additive}. Ainsi en prenant le supremum sur $\mathcal{M}(X,f)$ on obtient 
\[
        \sup_{\mu \in \mathcal{M}_{e}(X,f)}\mu(A) = \sup_{\mu \in \mathcal{M}(X,f)}\mu(A) \ge \mathrm{ocap}(A)
,\] ce qui conclut la preuve.
\end{proof}

\medskip

        Dans le cas où $A$ n'est pas fermé on peut étudier la dynamique du cercle $\mathbf{T}^{1}$ donnée par
        \begin{align*}
                R_{\alpha} : \mathbf{T}^{1} &\longmapsto \mathbf{T}^{1} \\ 
                        x &\longmapsto x + \alpha, \quad \alpha \in \R \setminus \Q
        ,\end{align*}
        que l'on sait uniquement ergodique pour la mesure de Lebesgue par le théorème d'équidistribution de Weyl.  

        Prenons un ensemble $A \defeq \{k\alpha, \; k\in \N\}$ non fermé, d'adhérence $\mathbf{T}^{1}$. Cet ensemble est dénombrable donc de mesure de Lebesgue nulle. En revanche \[
                \sup_{x\in \mathbf{T}^{1}}\#\{0\le k < n, \; T^{k}x \in A\} = n-1 
        ,\] comme $T^{k}(0) = k\alpha \in A, \; \; \forall k \in \N$. 
        Ainsi \[
                \mathrm{ocap}(A) = \lim_{N\to \infty}\frac{N-1}{N} = 1
        .\] On a trouvé un contre exemple dans le cas $A$ non fermé. 
\end{document}
