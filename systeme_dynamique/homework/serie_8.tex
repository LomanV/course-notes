\documentclass[12pt]{article}
\usepackage[top=1in, bottom=1in, left=1in, right=1in]{geometry}

\usepackage[onehalfspacing]{setspace}

\usepackage{amsmath, amssymb, amsthm}
\usepackage{enumerate, enumitem}
\usepackage{fancyhdr, graphicx, proof, comment, multicol}
\usepackage[none]{hyphenat}
\usepackage{dirtytalk}
\binoppenalty=\maxdimen
\relpenalty=\maxdimen

\usepackage{microtype}
\usepackage{mathpazo}
\usepackage{mdframed}
\usepackage{parskip}
\linespread{1.1}
\usepackage{graphicx}
\usepackage{subfig}

\usepackage{mathrsfs}
\usepackage{amsfonts}
\usepackage{amsmath}
\usepackage{amssymb}

\usepackage{mathtools}
\newcommand{\defeq}{\vcentcolon=}
\newcommand{\eqdef}{=\vcentcolon}

\newenvironment{statement}[1]
{\begin{mdframed}[linewidth=0.6pt]
        \textsc{Statement #1:}

}

\newenvironment{ex}[1]
{\begin{mdframed}[linewidth=0.6pt]
        \textsc{Exercice #1:}

}
    {\end{mdframed}}

\newcommand{\R}{\mathbf{R}}
\newcommand{\C}{\mathbf{C}}
\newcommand{\Z}{\mathbf{Z}}
\newcommand{\N}{\mathbf{N}}
\newcommand{\Q}{\mathbf{Q}}

\newcommand{\de}{\mathrm{d}}

\begin{document}
        \noindent
\textbf{Systèmes Dynamiques} \hfill \textbf{Vezin Lomàn}\\
\normalsize MAT551 \hfill Date de rendu: 30/11/2020\\

\begin{center}
\textbf{Exercice à rendre}
\end{center}

\bigskip
\hrule
\bigskip

\textbf{1.} On cherche tout d'abord à expliciter l'application $f$ affine par morceau sur le triangle ABC donné. Sur chaque sous triangle AOC, BOC elle s'écrit  \[
        f(x,y) = (a_{1}x + b_{1}y +c_{1}, a_{2}x + b_{2}y +c_{2}), \quad a_{i}, b_{i}, c_{i} \in \R
.\] Les conditions en A, B, O et C donnent finalement \[
        f(x,y) = \begin{cases}
                (2x-y+1, y) \quad &\text{si} \; x \le 0 \\
                (-2x-y+1,y) \quad &\text{si} \; x > 0
        \end{cases} 
.\]

On remarque que $f$ restreinte à chacune des lignes horizontales du triangle \[
L_{a} \defeq \{(x,a) \;|\; a \in [0,1], \; x \in [a-1,-a+1]\}
.\] s'identifie naturellement à l'application $T_{2}$. En effet par définition de $f$ on a 
\[
        \begin{cases}
                f([a-1,0]) &= [a-1,1-a] \\
                f([0,1-a]) &= [a-1,1-a]
        \end{cases}
.\] 
        
On constate de plus que les lignes horizontales sont les ensembles $f-$invariants du triangle. Pour une mesure ergodique $\mu$ sur ce dernier, elles sont donc de mesure 0 ou 1. Lorsque $\mu$ charge une ligne horizontale toutes les autres sont de mesure nulle puisqu'elles sont disjointes.

Ainsi l'application affine qui envoie la ligne horizontale chargée positivement par $\mu$ sur le segment $[0,1]$ et le reste du triangle sur 0 est une bijection mesurable de ABC dans le segment à un ensemble de mesure nulle près. Si on pose $\nu$ la mesure tirée en arrière, $f-$invariante, on obtient l'isomorphisme voulu.

\bigskip

\textbf{2.} Soit $\mu$ une mesure ergodique sur le triangle, par la question \textbf{1.} il existe une mesure invariante $\nu$ sur $[0,1]$ telle que les systèmes $(ABC,f,\mathcal{B},\mu)$ et $([0,1],T_{2},\mathcal{B},\nu)$ soient conjugués, c'est à dire en particulier \[
        h(\nu) = h(\mu)
.\]

Le même isomorphisme envoie dans l'autre sens une mesure $\nu$ de l'intervalle, $T_{2}-$invariante, sur une mesure $\mu$ $f-$invariante pour laquelle tout ensemble $f-$invariant est de mesure nulle ou totale, $\mu$ est donc ergodique.

Par le principe variationnel on obtient donc \[
        h_{top}(ABC,f) = \sup_{\mu \in \mathcal{M}_{e}(ABC,f)}h(\mu) = \sup_{\nu \in \mathcal{M}([0,1], T_{2})}h(\nu) = h_{top}([0,1],f)
.\]

Or nous savons par un résultat de PC que $h_{top}([0,1], T_{2}) = \log(2)$, ainsi nous en déduisons \[
        h_{top}(ABC,f) = \log(2)
.\] 

\bigskip

\textbf{3.} Soit $a \in [0,1]$ et $L_{a}$ la ligne horizontale associée dans le triangle ABC. Nous savons que l'application tente est expansive, par un théorème du cours le système $([0,1], T_{2})$ admet donc au moins une mesure invariante d'entropie maximale. Notons la $\nu$, alors \[
        h_{top}([0,1], T_{2}) = h(\nu)
.\]  L'inverse de l'isomorphisme qui envoie la $L_{a}$ sur $[0,1]$ et le reste du triangle sur 0 de la question \textbf{1.} permet d'obtenir une mesure tirée en arrière $\mu_{a}$ sur le triangle $f-$invariante. Puisque l'entropie est un invariant par isomorphisme nous avons  \[
        h(\nu) = h(\mu_{a})
.\] Comme nous avons montré à la question \textbf{2.} que $h_{top}(ABC,f) = h_{top}([0,1],T_{2})$ nous obtenons \[
h(\mu_{a}) = h_{top}(ABC,f)
,\] et $\mu_{a}$ est une mesure d'entropie maximale. Pour chaque ordonné $a \in [0,1]$ la mesure $\mu_{a}$ charge une ligne $L_{a}$ différente, elles sont donc toutes distinctes. Nous avons ainsi trouvé une infinité non dénombrable de mesures d'entropie maximale.
\bigskip

\textbf{4.} Le système $(ABC,f)$ n'est pas expansif. Fixons un recouvrement ouvert $\mathcal{U}$ quelconque du triangle. Puisque le triangle est compact, il existe par le lemme de Lebesgue un $\varepsilon > 0$ tel qu'au moins un élément de $\mathcal{U}$ contient une carré de côté $\varepsilon$. Comme $\mathcal{U}$ est fini, quitte à réduire $\varepsilon$ on peut supposer qu'un tel carré est d'intersection vide avec les autres ouverts de $\mathcal{U}$. Par définition de $f$, les points du triangle se déplacent uniquement horizontalement. Ainsi pour tout $n \in \N$ au moins un ouvert du recouvrement itéré  \[
\bigvee_{0\le k \le n}f^{-k}\mathcal{U}
\] contient au moins un segment vertical de longueur $\varepsilon > 0$. Le diamètre du recouvrement itéré $n$ fois vaut donc au moins $\varepsilon$ et ne peut en particulier pas tendre vers 0. Comme  $\mathcal{U}$ était choisi arbitrairement on en déduit que le système ne peut pas être expansif.

\medskip

Nous venons en particulier de trouver un système non expansif avec au moins une mesure d'entropie maximale, la réciproque du théorème du cours ne s'applique donc pas.
\end{document}
