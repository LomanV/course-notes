\documentclass[12pt]{article}
\usepackage[top=1in, bottom=1in, left=1in, right=1in]{geometry}

\usepackage[onehalfspacing]{setspace}

\usepackage{amsmath, amssymb, amsthm}
\usepackage{enumerate, enumitem}
\usepackage{fancyhdr, graphicx, proof, comment, multicol}
\usepackage[none]{hyphenat}
\usepackage{dirtytalk}
\binoppenalty=\maxdimen
\relpenalty=\maxdimen

\usepackage{microtype}
\usepackage{mathpazo}
\usepackage{mdframed}
\usepackage{parskip}
\linespread{1.1}
\usepackage{graphicx}
\usepackage{subfig}

\usepackage{mathrsfs}
\usepackage{amsfonts}
\usepackage{amsmath}
\usepackage{amssymb}

\usepackage{mathtools}
\newcommand{\defeq}{\vcentcolon=}
\newcommand{\eqdef}{=\vcentcolon}

\newenvironment{ex}[1]
{\begin{mdframed}[linewidth=0.6pt]
        \textsc{Exercice #1:}

}
    {\end{mdframed}}

\newcommand{\R}{\mathbf{R}}
\newcommand{\C}{\mathbf{C}}
\newcommand{\Z}{\mathbf{Z}}
\newcommand{\N}{\mathbf{N}}
\newcommand{\Q}{\mathbf{Q}}

\newcommand{\de}{\mathrm{d}}

\newcommand{\interior}[1]{%
  {\kern0pt#1}^{\mathrm{o}}%
}

\begin{document}
        \noindent
\textbf{Systèmes dynamiques} \hfill \textbf{Vezin Lomàn}\\
\normalsize MAT551  \hfill Date de rendu: 13/11/2020\\

\begin{center}
\textbf{Devoir maison}
\end{center}

\begin{ex}{1}
        Cobords et mesures invariantes
\end{ex}

\textbf{1.} Remarquons dans un premier temps l'inclusion
\[
        \mathcal{C}_{b}(X,T) \subset \mathcal{C}_{m}(X,T)
,\] en effet si l'on prend $\phi \in \mathcal{C}_{b}(X,T)$ avec fonction de transfert $\psi \in \mathcal{C}(X)$, alors pour toute mesure $\mu \in \mathcal{M}(X,T)$ 
\begin{align*}
        \int\phi\de\mu &= \int\psi\circ T - \psi\de\mu \\
                       &= \int \psi \circ T \de \mu - \int\psi\de\mu \\
                       &= 0 \quad \text{par } T \; \text{invariance de } \mu
.\end{align*}

En montrant que l'ensemble $\mathcal{C}_{m}(X,T)$ est un fermé on obtiendra le résultat souhaité \[
        \overline{\mathcal{C}_{b}(X,T)} \subset \overline{\mathcal{C}_{m}(X,T)} = \mathcal{C}_{m}(X,T)
.\] 
Comme $\mathcal{C}(X)$ est un espace métrique on peut utiliser un argument séquentiel. Soit $(\phi_{n})_{n\in\N}$ une suite de $\mathcal{C}_{m}(X,T)$ convergeant vers $\phi$. Alors pour $n \in \N$
\begin{align*}
        |\int\phi\de\mu| &= |\int\phi\de\mu - \int\phi_{n}\de\mu| \\
                         &\le \int|\phi-\phi_{n}|\de\mu \\
                         &\le\int\|\phi-\phi_{n}\|\de\mu \underset{n\to \infty}{\longrightarrow} 0
.\end{align*}
La limite appartient donc à $\mathcal{C}_{m}(X,T)$ qui est donc bien fermé.

\textbf{2.} \textit{i.} On applique un corollaire du théorème de Hahn Banach dans sa formulation géométrique, étant donné un espace vectoriel $E$, un sous espace vectoriel $M \subset E$ non dense, il existe une forme linéaire continue $\Lambda$ non nulle qui s'annule sur $M$. Il est facile de vérifier que $\mathcal{C}_{b}(X,T)$ et $\mathcal{C}_{m}(X,T)$ sont des $\R$ espaces vectoriels, le premier sous espace du second. Ils contiennent la fonction nulle sont stables par somme et multiplication par un scalaire réel. 

En supposant que l'inclusion $\overline{\mathcal{C}_{b}(X,T)} \subsetneq \mathcal{C}_{m}(X,T)$ soit stricte on obtient que $\mathcal{C}_{b}(X,T)$ n'est pas dense dans $\mathcal{C}_{m}(X,T)$. On conclut par le corollaire précédent l'existence de $\Lambda$ telle que  \[
        \mathcal{C}_{b}(X,T) \subset \{\phi \in \mathcal{C}(X) \;|\; \Lambda(\phi) = 0\} \subsetneq \mathcal{C}_{m}(X,T) 
.\] 
On sait de plus que le noyau d'une forme linéaire continue est fermé, ainsi on obtient \[
        \overline{\mathcal{C}_{b}(X,T)} \subset \{\phi \in \mathcal{C}(X) \;|\; \Lambda(\phi) = 0\} \subsetneq \mathcal{C}_{m}(X,T) 
.\]  
\textit{ii.} Puisque $\Lambda$ s'annule sur les cobords, pour toute fonction continue $\psi$ on a  \[
\int\psi\circ T -\psi\de\mu = 0 = \int\psi\circ T\de\mu - \int\psi\de\mu
.\] On en déduit que $\mu$ est $T-$invariante. Les mesures boréliennes positives $T_{\star}\mu_{+}, T_{\star}\mu_{-}$ satisfont donc
\begin{align*}
        \int\psi\de T_{\star}\mu_{+} - \int\psi\de T_{\star}\mu_{-} &= \int\psi \circ T\de\mu_{+} - \int\psi \circ T\de\mu_{-} \\ 
                                                                    &= \int\psi\circ T\de\mu \\
                                                                    &= \int\psi\de\mu \\
                                                                    &= \int\psi\de\mu_{+} - \int\psi\de\mu_{-} 
.\end{align*}
De l'hypothèse d'unicité on en déduit que pour tout borélien $A$, \[
        T_{\star}\mu_{+}(A) \ge \mu_{+}(A) \quad \text{et} \quad T_{\star}\mu_{-}(A) \ge \mu_{-}(A)
.\] 

Remarquons que $T_{\star}\mu_{+}, T_{\star}\mu_{-}$ sont aussi mutuellement séparées. Si $\mu_{+}(E) = 1, \; \mu_{-}(E) = 0$,
\[
        T_{\star}\mu_{+}(E) \ge \mu_{+}(E) = 1 \quad \text{et}
,\] 
\begin{align*}
        T_{\star}\mu_{-}(E) &= \int_{E}T\de\mu_{-} \\
                            &\le \|T\|_{\infty}\mu_{-}(E) \\
                            &= 0
,\end{align*} puisque $T$ est continue sur $X$ compact donc bornée.

En inversant les rôles de $\mu$ et $T_{\star}\mu$ qui sont égales, on obtient l'inégalité inverse et donc l'égalité. Par définition, nous venons de montrer que $\mu_{+}, \mu_{-}$ sont $T$-invariantes.

\medskip

\textit{iii.} Puisque $\mu_{+}, \mu_{-}$ sont $T-$invariantes et finies les mesures $\frac{1}{\mu_{+}(X)}\mu_{+}, \frac{1}{\mu_{-}(X)}\mu_{-} \in \mathcal{M}(X,T)$. Si $\phi \in \mathcal{C}_{m}(X,T)$ alors \[
        \frac{1}{\mu_{+}(X)}\int\phi\de\mu_{+} = 0 = \frac{1}{\mu_{-}(X)}\int\phi\de\mu_{-}
.\] Donc \[
\int\phi\de\mu_{+} = 0 = \int\phi\de\mu_{-}
,\] et il en découle que \[
\Lambda(\phi) = \int\phi\de\mu_{+} - \int\phi\de\mu_{-} = 0
.\]  
Ainsi nous obtenons une contradiction \[
        \mathcal{C}_{m}(X,T) \subset \{\phi \in \mathcal{C}(X) \;|\; \Lambda(\phi) = 0\} 
.\] 

\bigskip

\textbf{3.} On raisonne par contraposée. Si pour toutes mesures ergodiques $\mu_1, \mu_2$ on a \[
        \int\phi\de\mu_1 = \int\phi\de\mu_2 = c \in \R
,\] $c < \infty$ comme $\phi$ est continue sur $X$ compact donc bornée. Alors pour $\mu \in \mathcal{M}(X,T)$, comme les mesures ergodiques sont les points extrémaux des mesures  $T$-invariantes on obtient par le théorème de Choquet l'existence d'une distribution $M_{\mu}$ sur $\mathcal{M}(X,T)$ supportée par les mesures ergodiques telle que \[
\int\phi\de\mu = \int(\int\phi\de\nu)\de M_{\mu}(\nu), \; \nu \; \text{ergodiques}
.\]  
Comme par hypothèse  \[
        \int\phi\de\nu = c \quad \forall \nu \in \mathcal{M}_{e}(X,T)
,\] on en déduit \[
\int\phi\de\mu = c \quad \forall \mu \in \mathcal{M}(X,T)
.\] 

Ainsi pour $\mu \in \mathcal{M}(X,T)$ \[
        \int(\phi - c)\de\mu = \int\phi\de\mu - c\mu(X) = c - c = 0
.\] 

Donc en posant $\psi = \phi - c$ on obtient $\phi = \psi + c, \; \psi \in \mathcal{C}_{m}(X,T)$. On a donc bien montré \[
        \phi \in \mathcal{C}_{m}(X,T) + \R
.\]  

\bigskip

\textbf{4.} Par le point précédent il existe deux mesures ergodiques $\mu_1, \mu_2$ satisfaisant \[
\int\phi\de\mu_1 \neq \int\phi\de\mu_2
.\] $\mathcal{M}(X,T)$ muni de la topologie faible-* est métrisable, comme les mesures périodiques sont denses dans les mesures ergodiques $\mu_1, \mu_2$ sont toutes deux limites de suites de mesures périodiques. Par définition de la topologie faible-* l'application \[
\mu \longmapsto \int\phi\de\mu
\] et continue, on peut donc trouver au moins deux mesures périodiques, une dans chaque suite, $\tilde{\mu_1}, \tilde{\mu_2}$ disons, satisfaisant \[
\int\phi\de\tilde{\mu_1} \neq \int\phi\de\tilde{\mu_2}
.\]  

\begin{ex}{2}
       Ensemble de divergence, quelques généralités 
\end{ex}

\textbf{6.} On montre que l'ensemble $\mathcal{B}(\phi)$ est $T$-invariant. Soit $x \in T^{-1}(\mathcal{B}(\phi))$, et soit $y \in \mathcal{B}(\phi)$ tel que $T(x) = y$. Alors pour $n \in \N$ fixé on a 
\begin{align*}
        \frac{1}{n}\sum_{k=0}^{n-1} \phi\circ T^{k}(x) &= \frac{\phi(x)}{n} + \frac{1}{n}\sum_{k=1}^{n-1} \phi\circ T^{k-1}(y) \\
                                                       &=\frac{\phi(x)}{n} + \frac{1}{n}\sum_{k=0}^{n-1} \phi\circ T^{k}(y) - \frac{\phi(T^{n}(x))}{n}
.\end{align*}

Comme $\phi$ est bornée, en prenant la limite inférieure on remarque que \[
        \liminf_{n\to \infty} \frac{1}{n}\sum_{k=0}^{n-1} \phi\circ T^{k}(x) = \liminf_{n\to \infty} \frac{1}{n}\sum_{k=0}^{n-1} \phi\circ T^{k}(y)
,\] de même pour la limite supérieure. Ainsi $x \in \mathcal{B}(\phi)$ qui est donc bien $T-$invariant. 

\medskip

Par le théorème ergodique ponctuel, si $\mu \in \mathcal{M}(X,T)$, alors pour $\mu$ presque tout  $x \in X$ on a la convergence de la suite 

\begin{equation}
        (\frac{1}{n}\sum_{k=0}^{n-1} \phi\circ T^{k}(x))_{n\in\N}.
\end{equation}
L'ensemble de divergence des sommes de Birkhoff est donc de $\mu$ mesure nulle. 

\medskip

De plus, par le théorème 4.11 des notes de cours lorsque $\mu$ est uniquement ergodique la convergence de la suite (1) ci dessus est uniforme vers une constante, on a donc convergence pour tout  $x \in X$ et l'ensemble de divergence $\mathcal{B}(\phi)$ est vide.

\bigskip

\textbf{7.} Par la question \textbf{2.} nous savons que $\mathcal{C}_{m}(X,T) = \overline{\mathcal{C}_{b}(X,T)}$. Prenons donc $\phi \in \mathcal{C}_{m}(X,T)$ et une suite de cobords $(\phi_{n})_{n\in\N}$ donc $\phi$ est la limite uniforme. Notons pour tout $n \in \N$ \[
        \phi_{n} = \psi_{n}\circ T - \psi_{n}, \quad \psi_{n} \in \mathcal{C}(X)
,\] nous obtenons alors pour $x \in X$ quelconque et $j\in\N$
\begin{align*}
        \frac{1}{n}\sum_{k=0}^{n-1} \phi_{j}\circ T^{k}(x) &= \frac{1}{n}\sum_{k=0}^{n-1} (\psi_{j}\circ T - \psi_{j})\circ T^{k}(x) \\
                                                            &= \frac{1}{n}(\psi_{j}\circ T^{n}-\psi_{j})(x) \overset{CVU}{\longrightarrow}_{n} 0
\end{align*} puisque $\psi_{j}$ est continue sur $X$ qui est compact, et donc bornée.

Par convergence uniforme, en prenant la limite $j \to \infty$ on obtient le résultat souhaité. Puisque le choix de $x \in X$ était arbitraire on a convergence partout et l'ensemble de divergence $\mathcal{B}(\phi)$ est vide.

\bigskip

\begin{ex}{3}
        Ensemble de divergence pour des dynamiques minimales 
\end{ex}

\textbf{8.} $(X,T)$ est un espace métrique nous pouvons appliquer un argument séquentiel. Soit $n \ge N$, la somme \[
\frac{1}{n}\sum_{k=0}^{n-1} \phi\circ T^{k}
\] est une application continue comme somme de compositions d'applications continues. Si l'on considère une suite $(x_{j})_{j\in\N}$ de $W(N, \varepsilon)$ convergeant vers $x \in X$, alors nous obtenons par passage à la limite et continuité 
\begin{alignat*}{3}
        \lim_{j\to \infty}\frac{1}{n}\sum_{k=0}^{n-1} \phi\circ T^{k}(x_{j}) &\ge \int\phi\de\mu + \varepsilon \quad &&\text{d'une part,} \\
                                                                             &= \frac{1}{n}\sum_{k=0}^{n-1} \phi\circ T^{k}(x) \quad &&\text{d'autre part}
.\end{alignat*}
Comme $n \ge N$ était arbitraire $x \in W(N,\varepsilon)$ qui est donc fermé.

\medskip

Si par l'absurde $W(N,\varepsilon)$ n'est pas d'intérieur vide il contient un ouvert non vide $U$. Comme $(X,T)$ est minimal il existe pour cet ouvert  $U$ un entier $J > 0$ tel que  \[
X = \bigcup_{0\le j \le J}T^{-j}U
.\] 

Alors comme $U \subset W(N,\varepsilon)$, pour tout $x \in X$ il existe $j \in \{0, \ldots, J\}$ tel que \[
        \frac{1}{n}\sum_{k=0}^{n-1} \phi\circ T^{k+j}(x) \ge \int\phi\de\mu + \varepsilon, \; \forall n \ge N
.\] 

De plus on a
\begin{align*}
        \frac{1}{n+j}\sum_{k=0}^{n+j-1} \phi\circ T^{k}(x) &= \frac{1}{n+j}\sum_{k=0}^{j-1} \phi\circ T^{k}(x) + \frac{n}{n+j}\frac{1}{n}\sum_{k=0}^{n-1} \phi\circ T^{k+j}(x) \\
                                                           &\ge \frac{1}{n+j}\sum_{k=0}^{j-1} \phi\circ T^{k}(x) + \frac{n}{n+j}(\int\phi\de\mu + \varepsilon)
.\end{align*}

On en déduit comme $j$ est fixe que \[
        \lim_{n}\frac{1}{n}\sum_{k=0}^{n-1} \phi\circ T^{k}(x) = \lim_{n}\frac{1}{n+j}\sum_{k=0}^{n+j-1} \phi\circ T^{k}(x) \ge \int\phi\de\mu + \varepsilon
.\] 

Mais alors \[
\lim_{n}\frac{1}{n}\sum_{k=0}^{n-1} \phi\circ T^{k}(x) > \int\phi\de\mu 
.\] $x \in X$ étant arbitraire et $\mu$ ergodique $X$ est donc de mesure nulle par le théorème de Birkhoff, ce qui est absurde. 

\bigskip

\textbf{9.} Nous venons de montrer que pour tous $N >0, \varepsilon > 0$ l'ensemble $W(N,\varepsilon)$ est maigre. En effet comme il est fermé nous avons  \[
        \overline{W(N,\varepsilon)} = W(N,\varepsilon) \quad \text{et donc} \quad \interior{(\overline{W(N,\varepsilon)})} = \interior{W(N,\varepsilon)} = \emptyset 
.\] 

Nous savons que le complémentaire d'un ensemble maigre contient un sous ensemble dense. Aussi
\begin{align*}
        W(N, \varepsilon) &= \{x\in X \;|\; \frac{1}{n}\sum_{k=0}^{n-1} \phi\circ T^{k}(x) \ge \int\phi\de\mu + \varepsilon, \; \forall n \ge N\} \\
                          &= \{x \in X \;|\; \liminf_{n} \frac{1}{n}\sum_{k=0}^{n-1} \phi\circ T^{k}(x) \ge \int\phi\de\mu + \varepsilon\}
.\end{align*}
Comme le résultat est valable pour tout $\varepsilon > 0$, que $\mu$ est une mesure ergodique et en remarquant que  \[
        \mathcal{B}(\phi)^{c} = \{x\in X \;|\; \liminf_{n}\frac{1}{n}\sum_{k=0}^{n-1} \phi\circ T^{k}(x) \ge \limsup_{n} \frac{1}{n}\sum_{k=0}^{n-1} \phi\circ T^{k}(x)\} 
,\] on en conclut que $\mathcal{B}(\phi)$ contient un sous ensemble dense de $X$. 

\bigskip

\begin{ex}{4}
       Ensemble de divergence pour le décalage sur $\{0,1\}^{\N}$ 
\end{ex}

\textbf{10.} Commençons par remarquer que \[
        h_{top}(\{0,1\}^{\N}) = \limsup_{n}\frac{1}{n}\#C_{n} = \log(2)
\] puisque $C_{n}$ est une partition de $\{0,1\}^{\N}$ de cardinal $2^{n}$. 

Considérons $L \subset \{0,1\}^{\N}$ le sous ensemble des suites dont l'orbite est dense dans $\{0,1\}^{\N}$. Par définition de $L$ et comme les $n$-cylindres sont des ouverts de $\{0,1\}^{\N}$ on obtient directement \[
        h_{top}(L) = \log(2)
.\] 

On montre que sur $L$ les sommes de Birkhoff convergent uniformément vers une constante. On en déduira le résultat puisque cela est équivalent à l'unique ergodicité du sous système $(L, \sigma)$.
\medskip

\textbf{11.} \textit{i.} Dans un premier temps remarquons que
\begin{align*}
        q_{2k+1}-q_{2k} &= |w_{1}|m_{k}+l_{2k} \\
        q_{2k+2}-q_{2k+1} &= |w_{2}|n_{k}+l_{2k+1}
.\end{align*}

Ainsi l'hypothèse $\frac{q_{2k+1}}{q_{2k}} \to_{k} \infty$ implique asymptotiquement que
\begin{align*}
        q_{2k+1} &\sim |w_{1}|m_{k}+l_{2k} \\
        q_{2k+2} &\sim |w_{2}|n_{k}+l_{2k+1}
.\end{align*}

Et donc
\begin{align*}
        \delta_{1} &= \lim_{k} \frac{l_{2k}}{q_{2k+1}} \\
        \delta_{2} &= \lim_{k} \frac{l_{2k+1}}{q_{2k+2}}
.\end{align*}

Soit $C_{q_{j}}$ l'ensemble des $q_{j}$ cylindres. Pour chaque mot $u_{2k+1}$ on a au moins $2^{l_{2k}}$ choix possibles correspondant au choix de la troncature $w(l_{2k}), \; w \in L$. De même pour les mots $u_{2k+2}$ on a au moins $2^{l_{2k+1}}$ choix. Cela nous donne les deux inégalités ci dessous
\begin{align*}
        \#\{C \in C_{q_{2k+1}} \;|\; C \cap K \neq \emptyset\} &\ge 2^{l_{2k}} \\
        \#\{C \in C_{q_{2k+2}} \;|\; C \cap K \neq \emptyset\} &\ge 2^{l_{2k+1}}
.\end{align*}

On en conclut en prenant le logarithme et par passage à la limite que \[
        h_{top}(K) = \limsup_{n}\frac{1}{n}\#\{C\in C_{n} \;|\; C \cap K \neq \emptyset\} \ge \max_{i}\delta_{i}\log(2)
.\] 

\medskip

\textit{ii.} Par le théorème de Tychonoff $\{0,1\}^{\N}$ est compact, $\phi$ continue sur $\{0,1\}^{\N}$ est donc uniformément continue. Soit $\varepsilon > 0$, par continuité uniforme de $\phi$ il existe un $\delta > 0$ tel que  \[
        d(x,y) < \delta \implies |\phi(x)-\phi(y)| < \varepsilon
.\]  Rappelons la définition de la métrique $d$ sur $\{0,1\}^{\N}$, \[
d(x,y) = \sum_{n\in\N} \frac{\delta_{x_{n},y_{n}}}{2^{n}}
.\] Sur un $r$-cylindre cette distance vaut au plus $\frac{1}{2^{r-1}}$. On peut donc trouver $r$ assez grand tel que $\frac{1}{2^{r-1}} < \delta$ ce qui donne le résultat.

\medskip

\textit{iii.}  Commençons par la première inégalité. Lorsque $n = q_{2k+1}$ pour un $k\in\N$, on obtient \[
        \frac{1}{n}\sum_{i=0}^{n-1} \phi\circ\sigma^{i}(u) = \frac{1}{q_{2k+1}}(\sum_{i=0}^{q_{2k}-1} \phi\circ\sigma^{i}(u) + \sum_{i=q_{2k}}^{q_{2k}+l_{2k}-1} \phi\circ\sigma^{i}(u) + \sum_{i=q_{2k}+l_{2k}}^{q_{2k+1}-1} \phi\circ\sigma^{i}(u))
.\] 

$\phi$ est bornée puisque continue sur $\{0,1\}^{\N}$ qui est compact. Ainsi si $M$ borne $\phi$ on peut majorer le premier terme  \[
        \frac{1}{q_{2k+1}}\sum_{i=0}^{q_{2k}-1} \phi\circ\sigma^{i}(u) \le \frac{q_{2k}}{q_{2k+1}}M \to_{k} 0
.\] puisque par hypothèse $\frac{q_{2k+1}}{q_{2k}} \to_{k} \infty$.

Regardons à present le deuxième terme, $i \in \{q_{2k}, \ldots, q_{2k} +l_{2k}-1\}$. Sauf sur les $m_{k}$ derniers termes on a par définition de $u \in K$ que $\sigma^{i}(u)$ coïncide avec $\sigma^{i - q_{2k}}(w)$ pour un $w \in L$. Soit $\varepsilon > 0$, puisque la suite $(m_{k})_{k}$ tend vers $+\infty$ on peut prendre $k$ suffisamment grand de sorte que par le point \textit{ii.}  \[
        |\phi\circ\sigma^{i}(u) - \phi\circ\sigma^{i-q_{2k}}(w)| < \varepsilon
.\]  
Ainsi on en déduit que 
\begin{align*}
        \frac{1}{q_{2k+1}}\sum_{i=q_{2k}}^{q_{2k}+l_{2k}-1} \phi\circ\sigma^{i}(u) &\le \frac{l_{2k}}{q_{2k+1}}\frac{1}{l_{2k}}(\sum_{i=0}^{l_{2k}-1} \phi\circ\sigma^{i}(w) + \varepsilon) 
.\end{align*}

Par convergence uniforme des sommes de Birkhoff sur $L$ et puisque le choix de $\varepsilon > 0$ est arbitraire, on obtient donc par passage à la limite $k \to \infty$ \[
\frac{l_{2k}}{q_{2k+1}}\frac{1}{l_{2k}}(\sum_{i=0}^{l_{2k}-1} \phi\circ\sigma^{i}(w) + \varepsilon) \to_{k} \delta_{1}\int\phi\de\mu_{\max} 
.\] 

On regarde enfin le dernier terme, $i \in \{q_{2k}+l_{2k}, \ldots, q_{2k+1}-1\}$. On a par définition de $u$ que $\sigma^{i}(u)$ coïncide avec $\sigma^{i-(q_{2k}+l_{2k})}(w_1)$. Ainsi on obtient
\begin{align*}
        \frac{1}{q_{2k+1}}\sum_{i=q_{2k}+l_{2k}}^{q_{2k+1}-1} \phi\circ\sigma^{i}(u) &\le \frac{|w_{1}|m_{k}}{q_{2k+1}}\frac{1}{|w_{1}|m_{k}}\sum_{i=0}^{|w_{1}|m_{k}-1} \phi\circ\sigma^{i}(w_{1}) \\
                                                                                     &\to_{k}(1-\delta_{1})\int\phi\de\mu_{1}
.\end{align*}
Par choix de la mesure $\mu_{1}$ et comme asymptotiquement $1-\frac{l_{2k}}{|w_{1}|m_{k}+l_{2k}} \sim \frac{|w_{1}|m_{k}}{q_{2k+1}}$.

\medskip

Pour $n \in \N$ quelconque il existe $k \in \N$ tel que $n \le q_{2k+1}$ comme $q_{2k+1}\to_{k} \infty$ et donc par les trois calculs précédents \[
        \liminf_{n}\frac{1}{n}\sum_{i=0}^{n-1} \le 0 + \delta_{1}\int\phi\de\mu_{\max} + (1-\delta_{1})\int\phi\de\mu_{1}
.\]  

\medskip

On montre à présent la deuxième inégalité de la question. On prend cette fois $n = q_{2k+2}$ et on applique exactement le même raisonnement.
Ainsi on sépare la somme comme précédemment \[
        \frac{1}{n}\sum_{i=0}^{n-1} \phi\circ\sigma^{i}(u) = \frac{1}{q_{2k+2}}(\sum_{i=0}^{q_{2k+1}-1} \phi\circ\sigma^{i}(u) + \sum_{i=q_{2k+1}}^{q_{2k+1}+l_{2k+1}-1} \phi\circ\sigma^{i}(u) + \sum_{i=q_{2k+1}+l_{2k+1}}^{q_{2k+2}-1} \phi\circ\sigma^{i}(u))
.\] 
Le premier terme tends vers 0, pour $\varepsilon > 0$, en prenant $k$ suffisamment grand le second terme satisfait l'inégalité \[
        \frac{1}{q_{2k+2}}\sum_{i=q_{2k+1}}^{q_{2k+1}+l_{2k+1}-1} \phi\circ\sigma^{i}(u) \ge \frac{l_{2k+1}}{q_{2k+2}}\frac{1}{l_{2k+1}}(\sum_{i=0}^{l_{2k+1}-1} \phi\circ\sigma^{i}(w) - \varepsilon) 
,\] avec $w \in L$. Ainsi par convergence uniforme des sommes de Birkhoff sur $L$, comme $\frac{l_{2k+1}}{q_{2k+2}}\to_{k}\delta_{2}$ et comme $\varepsilon > 0$ est arbitraire, on obtient  \[
\frac{l_{2k+1}}{q_{2k+2}}\frac{1}{l_{2k+1}}(\sum_{i=0}^{l_{2k+1}-1} \phi\circ\sigma^{i}(w) + \varepsilon) \to_{k} \delta_{2}\int\phi\de\mu_{\max} 
.\] 

Enfin pour le dernier terme comme cette fois $\sigma^{i}(u)$ coïncide avec $\sigma^{i-(q_{2k+1}+l_{2k+1})}(w_{2})$ on obtient l'inégalité
\begin{align*}
        \frac{1}{q_{2k+2}}\sum_{i=q_{2k+1}+l_{2k+1}}^{q_{2k+2}-1} \phi\circ\sigma^{i}(u) &\ge \frac{|w_{2}|n_{k}}{q_{2k+2}}\frac{1}{|w_{2}|n_{k}}\sum_{i=0}^{|w_{2}|n_{k}-1} \phi\circ\sigma^{i}(w_{2}) \\
                                                                                     &\to_{k}(1-\delta_{2})\int\phi\de\mu_{2}
,\end{align*} par choix de la mesure $\mu_{2}$.

Pour $n \in \N$ quelconque comme $q_{2k+2}\to_{k}\infty$ il existe $k \in \N$ tel que $n \le q_{2k+2}$. Ainsi en prenant le supremum sur $n$ et par les trois calculs précédents on obtient cette fois  \[
        \limsup_{n}\frac{1}{n}\sum_{i=0}^{n-1} \ge 0 + \delta_{2}\int\phi\de\mu_{\max} + (1-\delta_{2})\int\phi\de\mu_{2}
.\]

Nous avons montré les deux inégalités désirées.
\medskip

\textit{iv.} Soit $\alpha > 0$, on peut trouver $l, m, n$ trois suites telles que  \[
\delta_{1} = 1 - \alpha = \delta_{2}
.\] 
Ainsi on obtient par le point \textit{i.} \[
h_{top}(K) \ge (1-\alpha)\log(2)
.\] De plus par la question \textbf{4.} on peut supposer quitte à changer le choix de $w_{1}, w_{2} \in L$, $\int\phi\de\mu_{1} < \int\phi\de\mu_{2}$. On obtient alors pour $u \in K$ en utilisant les inégalités de \textit{iii.}
\[
        \liminf_{n}\frac{1}{n}\sum_{k=0}^{n-1} \phi\circ\sigma^{k}(u) < \limsup_{n}\frac{1}{n}\sum_{k=0}^{n-1} \phi\circ\sigma^{k}(u)
.\] 

\medskip

\textbf{12.} Par le point \textit{iv.} de la question précédente $K \subset \mathcal{B}(\phi)$. Ainsi \[
        h_{top}(\mathcal{B}(\phi)) \ge h_{top}(K)
.\] Toujours par le point \textit{iv.} on a de plus  \[
h_{top}(K) \ge (1-\alpha)\log(2), \quad \forall \alpha > 0
.\]
Par passage à la limite $\alpha \to 0$, on obtient donc \[
        h_{top}(\mathcal{B}(\phi)) \ge \log(2)
.\] 
Nous avons remarqué dans la question \textbf{10.} \[
        h_{top}(\{0,1\}^{\N}) = \log(2)
,\] et cela force donc \[
h_{top}(\mathcal{B}(\phi)) = \log(2)
.\]  
\end{document}
